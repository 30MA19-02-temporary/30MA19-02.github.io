% !TeX root = main.tex
% !TeX spellcheck = en_US
% !TeX encoding = utf8
% !TeX program = pdflatex
% !TEX options = -file-line-error -interaction=nonstopmode --shell-escape %DOC%
% !BIB program = biblatex
% !BIB options = %DOCFILE%

% https://apastyle.apa.org/style-grammar-guidelines
% https://www.ams.org/publications/authors/AMS-StyleGuide-online.pdf
\documentclass[stu, babel, american, biblatex, a4paper, leqno, draftall]{apa7}
\usepackage{csquotes}
\addbibresource{sections/bibliography.bib}
\DeclareLanguageMapping{american}{american-apa}
\usepackage{amsmath, amsfonts, amssymb, amsthm}
\usepackage{mathtools}
\usepackage{thmtools}
\usepackage{tensor}
\usepackage{braket}
\usepackage{hyperref}
\usepackage[capitalise, sort&compress, noabbrev, nameinlink]{cleveref}
\usepackage[mode=buildnew]{standalone}
\usepackage{pgfplots}
\usepackage{subfiles}

\allowdisplaybreaks
\crefname{paragraph}{paragraph}{paragraphs}
\Crefname{paragraph}{Paragraph}{Paragraphs}
\crefname{subparagraph}{subparagraph}{subparagraphs}
\Crefname{subparagraph}{Subparagraph}{Subparagraphs}
\pgfplotsset{compat=newest}
\usepgfplotslibrary{external}
\tikzexternalize


\title{Prove of Constant Radius Model}
\shorttitle{30MA19-02}
\leftheader{Hanchai, Sakepisit}
\authorsnames{Hanchai Nonprasart, Sakepisit Maysamat}
\authorsaffiliations{Mahidol Wittayanusorn School}
\course{SCI 30196: Science Project 1}
\professor{Amornsri Amornvatcharapong, Teerapong Suksumran}
\duedate{\today} % Last edit day when the draft watermark is removed.
\abstract{This is the first generation of the model. It's a draft. Who would even care to read the abstract when it's not done anyways?}
\keywords{None 1, None 2, None 3}

%!TEX root = ../main.tex

% "well-known" notations
\newcommand{\N}{\mathbb{N}}
\newcommand{\R}{\mathbb{R}}
\newcommand{\inner}[2]{\left\langle{#1},{#2}\right\rangle}
\newcommand{\lie}[2]{\left[{#1},{#2}\right]}
\newcommand{\norm}[1]{\left|\left|{#1}\right|\right|}
\DeclareMathOperator{\sign}{sgn}
\DeclarePairedDelimiter\abs{\lvert}{\rvert}
\DeclareMathOperator{\contr}{contr}
\DeclareMathOperator{\diag}{diag}

% notation which may required some prelimitaries or definition
\newcommand{\range}[2]{\Set{#1..#2}}

% placeholder commands - Use to find it easier with Ctrl+F
\newcommand{\proofof}[1]{Proof of \cref{#1}} % Proof name
\newcommand{\toshowthat}[1]{To show that #1} % Subproof name
\newenvironment{subproof}[1]{%
    \renewcommand{\qedsymbol}{}
    \begin{proof}[To show that #1]\( \)\par\nobreak\ignorespaces
        }{%
    \end{proof}
}

% defined notations
\newcommand{\elements}{\mathbb{M}}
\newcommand{\groupoperation}[1]{\otimes_{#1}}
\newcommand{\points}[1]{\mathbb{P}_{#1}}
\newcommand{\transformations}[1]{\mathbb{T}_{#1}}
\newcommand{\group}[1]{\mathbb{G}_{#1}}
\newcommand{\subgroup}[1]{\mathbb{H}_{#1}}
\newcommand{\charts}[1]{\varphi_{#1}}
\newcommand{\innerprod}[1]{g_{#1}}
%!TEX root = ../main.tex

\declaretheorem[
    name=Definition,
    style=definition,
    refname={definition,definitions},
    Refname={Definition,Definitions},
]{definition}
\declaretheorem[
    name=Example,
    parent=definition,
    style=definition,
    refname={example,examples},
    Refname={Example,Examples},
]{example}
\declaretheorem[
    name=Preposition,
    style=plain,
    refname={proposition,propositions},
    Refname={Proposition,Propositions},
]{proposition}
\declaretheorem[
    name=Theorem,
    style=plain,
    refname={theorem,theorems},
    Refname={Theorem,Theorems},
]{theorem}
\declaretheorem[
    name=Lemma,
    style=plain,
    refname={lemma,lemmas},
    Refname={Lemma,Lemmas},
]{lemma}
\declaretheorem[
    name=Corollary,
    style=plain,
    refname={corollary,corollaries},
    Refname={Corollary,Corollaries},
]{corollary}
\declaretheorem[
    name=Conjecture,
    style=plain,
    refname={conjecture,conjectures},
    Refname={Conjecture,Conjectures},
]{conjecture}
\declaretheorem[
    name=Remark,
    style=remark,
    refname={remark,remarks},
    Refname={Remark,Remarks},
]{remark}


\newcounter{Counter}

\theoremstyle{definition}
\newtheorem{ModelGroupElement}{Definition}[Counter]
\newtheorem{ModelMetric}[ModelGroupElement]{Definition}

\theoremstyle{plain}
\newtheorem{ModelGroupAssertion}{Assertion}[Counter]
\newtheorem{ModelCurvatureAssertion}[ModelGroupAssertion]{Assertion}

\begin{document}
\maketitle
\tableofcontents

\section{Rewrite note}
This proof will be organized using the following guideline
\begin{APAenumerate}
    \item reviewing existing definitions and theorem
    \item proving certain property of the existing mathematical objects
    \item defining the conditions of the model
    \item formulating of the model
    \item parameterize the model
    \item asserting defined property of the model
\end{APAenumerate}
It turned out that figures will crash overleaf, if you want to view such file, please compile them seperately.
To merge the file, it is needed to either pay the subscription or download and run it on your computer.

\subfile{sections/preliminary.tex}
\subfile{sections/objective.tex}
\subfile{sections/foundation.tex}
\subfile{sections/parameterization.tex}
\subfile{sections/geometry.tex}
\subfile{sections/transformation.tex}
\subfile{sections/assertion.tex}

\section{Future plan}
\subsection{Model II}
It is very simple to be able to model composite geometries e.g. \(S^2 \times E\) by tensor product of the existing model. But to be able to merge them as smooth model may be challenging since not all combination of basis curvature have their own intrinsic geometry. So it may be to find independent variable for each basis or to introduce extrinsic curvature (Model A).
\subsection{Model B}
It is known that \(E\) emerged at \(n\ge1\) while \(S\) and \(H\) emerged at \(n\ge2\) and there's more complex pure geometries than these that emerged in higher dimension. It is interesting and challenging to explore such geometries and prove whether the curvature still works as indicator in such geometries or are there any patterns for their symmetries.
\subsection{Model A}
This model is based on curvature and mostly just 3 basis geometries and extrinsic curvature which seems to be interesting despite some critical result in some combination e.g. \(S^1 \times S^1\) vs \(S^2\). It can be even more challenging to have variable curvature with respect to other intrinsic position.
\section*{}
\printbibliography
\subfile{sections/figures.tex}
\end{document}
