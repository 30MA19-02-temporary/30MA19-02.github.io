% !TeX root = ../../main.tex
\documentclass[../methodology.tex]{subfiles}
\begin{document}
\subsection{Geometric transformation}
\begin{lemma}\label{M:transform:distance}
    For any tangent vector \(p,q\in T_p N\)
    and point matrix \(X\),
    \begin{equation*}
        \left\langle Xp, Xq\right\rangle = \left\langle p, q\right\rangle
        \text{.}
    \end{equation*}
    \[
    \]
\end{lemma}
\begin{proof}[Proof of \cref{M:transform:distance}]
    If \(\lambda>0\), it can be proved that \(X\) is orthogonal.
    Therefore,
    \begin{align*}
        X^{-1} & =X^T           \\
        X^{-1} & =X^TI          \\
        I      & =X^TIX         \\
        g      & =X^TgX\text{.}
    \end{align*}
    If \(\lambda<0\), it can be proved that \(X\) is \((1,n)\)-indefinite orthogonal.
    Therefore,
    \begin{align*}
        X^{-1} & =gX^Tg         \\
        I      & =gX^TgX        \\
        g^{-1} & =X^TgX         \\
        g      & =X^TgX\text{.}
    \end{align*}
    If \(\lambda\approx 0\), it can be proved that \(X\) is an identity and the same argument applies.
    Since \(g=X^TgX\),
    \begin{align*}
        \left\langle Xp, Xq\right\rangle
         & = \left(Xp\right)^Tg\left(Xq\right)      \\
         & = p^TX^TgXq                              \\
         & = p^Tgq                                  \\
        \left\langle Xp, Xq\right\rangle
         & = \left\langle p, q\right\rangle\text{.}
        \qedhere
    \end{align*}
\end{proof}
\end{document}