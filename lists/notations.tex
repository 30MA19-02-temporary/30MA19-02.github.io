%!TEX root = main.tex

% "well-known" notations
\newcommand{\N}{\mathbb{N}}
\newcommand{\R}{\mathbb{R}}
\newcommand{\inner}[2]{\left\langle{#1},{#2}\right\rangle}
\newcommand{\norm}[1]{\left|\left|{#1}\right|\right|}
\DeclareMathOperator{\sign}{sgn}
\DeclareMathOperator{\abs}{abs}
\DeclareMathOperator{\contr}{contr}
\DeclareMathOperator{\diag}{diag}

% Notation used without definition
\newcommand{\defeq}{\vcentcolon=} % To declare equation as define so there's no need to prove

% notation which may required some prelimitaries or definition
\DeclareMathOperator{\ortho}{O}
\DeclareMathOperator{\genlin}{GL}
\DeclareMathOperator{\euclid}{Euc}
\newcommand{\range}[2]{\set{#1..#2}}

% placeholder commands - Use to find it easier with Ctrl+F
\newcommand{\skipped}{The proof is left as an exercise to the other author.}  % pending proof - JUST DO IT ALREADY
\newcommand{\citenone}{*Not Cited*}  % pending citation - JUST DO IT ALREADY
\newcommand{\proto}[1]{#1...}  % pending operation
\newcommand{\proofof}[1]{Proof of \cref{#1}} % Proof name
\newcommand{\toshowthat}[1]{To show that #1} % Subproof name
\newenvironment{subproof}[1]{%
    \renewcommand{\qedsymbol}{}
    \begin{proof}[To show that #1]$ $\par\nobreak\ignorespaces
        }{%
    \end{proof}
}

% defined notations
\newcommand{\elements}{\mathbb{M}}
\newcommand{\groupoperation}[1]{\otimes_{#1}}
\newcommand{\points}[1]{\mathbb{P}_{#1}}
\newcommand{\transformations}[1]{\mathbb{T}_{#1}}
\newcommand{\group}[1]{\mathbb{G}_{#1}}
\newcommand{\subgroup}[1]{\mathbb{H}_{#1}}
\newcommand{\charts}[1]{\varphi_{#1}}
\newcommand{\innerprod}[1]{g_{#1}}