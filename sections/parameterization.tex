% !TeX root = ../main.tex
\documentclass[../main.tex]{subfiles}
\begin{document}
\section{Model Parametrization}
\begin{definition}\label{M:Parameter}
For any point matrix $\tensor{X^\pm_{\lambda,n}\left(\theta,\phi_1,\phi_2,\dots,\phi_n\right)}{}$,
$n$-dimensional vector $\tensor{\theta}{}$
is defined as\textit{position parameter}.
\end{definition}
\begin{definition}\label{M:Vector}
For point matrix $\tensor{X}{}$,
$\left(n+1\right)$-dimensional column vector $\tensor{p}{}\coloneqq\frac{1}{\lambda}\tensor{X}{}\cdot\tensor{e}{^1}=\frac{1}{\lambda}\tensor{X}{_1}$
is defined as\textit{position vector}.
\end{definition}
\begin{definition}\label{M:Vector:Set}
$P^\ast\left(\lambda,n\right)$ is a set of position vectors.
\end{definition}
\begin{lemma}\label{M:Vector:Position}
For point matrix $X=PO$
where $P$ and $O$ are position and orientation matrix respectively,
$\tensor{p}{}=\frac{1}{\lambda}\tensor{X}{_1}=\frac{1}{\lambda}\tensor{P}{_1}$.
\end{lemma}
\begin{proof}[\proofof{M:Vector:Position}]
\begin{align*}
\tensor{p}{^i}
&=\frac{1}{\lambda}\tensor{X}{^i_1}&&\text{\cref{M:Vector}}\\
&=\frac{1}{\lambda}\sum_j{\tensor{P}{^i_j}\tensor{O}{^j_1}}&&\text{\cref{Matrix:Product}}\\
&=\frac{1}{\lambda}\tensor{P}{^i_1}&&\text{\cref{M:Orientation}}\\
\tensor{p}{}
&=\frac{1}{\lambda}\tensor{P}{_1}&&\qedhere
\end{align*}
\end{proof}
\begin{lemma}\label{M:Vector:Value}
Given position parameter $\tensor{\theta}{}$, position vector can be evaluated as the following.
\begin{equation*}
\psi_0^{-1}\colon\tensor{\theta}{}\mapsto\tensor{p}{}
=\frac{1}{\lambda}
\begin{pmatrix}
\prod_{j\in\Set{1..n}}{\cos_\lambda{\tensor{\theta}{^j}}}\\
\sin_\lambda{\tensor{\theta}{^{i-1}}}\prod_{j\in\Set{i..n}}{\cos_\lambda{\tensor{\theta}{^j}}}\\
\sin_\lambda{{\tensor{\theta}{^n}}}\\
\end{pmatrix}\text{.}
\end{equation*}
\end{lemma}
\begin{proof}[\proofof{M:Vector:Value}]
Simplify \cref{M:Vector:Position,M:Position}.
\end{proof}
\begin{lemma}\label{M:Parameter:Value}
Given position vector $\tensor{p}{}$, position parameter can be calculated as the following.
\begin{align*}
\psi_0
&\colon\tensor{p}{}\mapsto\tensor{\theta}{}
=
\begin{pmatrix}
\arcsin_\lambda^{\sign{\frac{1}{\lambda}\tensor{p}{^1}}}{\frac{\tensor{p}{^2}}{\prod_{j\in\Set{2..n}}{\cos_\lambda{\tensor{\theta}{^j}}}}}\\
\arcsin_\lambda{\frac{\frac{1}{\lambda}\tensor{p}{^{i+1}}}{\prod_{j\in\Set{i+1..n}}{\cos_\lambda{\tensor{\theta}{^j}}}}}\\
\arcsin_\lambda{\frac{1}{\lambda}\tensor{p}{^{n+1}}}\\
\end{pmatrix}\\
&\in
\begin{cases}
P\to\left(-\frac{\pi}{\lambda},\frac{\pi}{\lambda}\right]\times\left[-\frac{1}{2}\frac{\pi}{\lambda},\frac{1}{2}\frac{\pi}{\lambda}\right]^{n-1}&\text{if $\lambda>0$}\\
P\to\R^{n}&\text{if $\lambda\le0$}\\
\end{cases}
\end{align*}
where $\cos_\lambda\left(\arcsin_\lambda^{\pm}\left(x\right)\right) =\pm\cos_\lambda\left(\arcsin_\lambda\left(x\right)\right)$.
\end{lemma}
\begin{proof}[\proofof{M:Parameter:Value}]
From \cref{M:Vector:Value},
\begin{align*}
\tensor{p}{}
&=\frac{1}{\lambda}
\begin{pmatrix}
\prod_{j\in\Set{1..n}}{\cos_\lambda{\tensor{\theta}{^j}}}\\
\sin_\lambda{\tensor{\theta}{^{i-1}}}\prod_{j\in\Set{i..n}}{\cos_\lambda{\tensor{\theta}{^j}}}\\
\sin_\lambda{{\tensor{\theta}{^n}}}\\
\end{pmatrix}\\
\frac{1}{\lambda}\sin_\lambda{{\tensor{\theta}{^n}}}&=\tensor{p}{^{n+1}}\\
\tensor{\theta}{^n}&=\arcsin_\lambda{\lambda\tensor{p}{^{n+1}}}\\
\frac{1}{\lambda}\sin_\lambda{\tensor{\theta}{^{i-1}}}\prod_{j\in\Set{i..n}}{\cos_\lambda{\tensor{\theta}{^j}}}&=\tensor{p}{^{i}}\\
\sin_\lambda{\tensor{\theta}{^{i-1}}}&=\frac{\lambda\tensor{p}{^{i}}}{\prod_{j\in\Set{i..n}}{\cos_\lambda{\tensor{\theta}{^j}}}}\\
\tensor{\theta}{^{i-1}}&=\arcsin_\lambda{\frac{\lambda\tensor{p}{^{i}}}{\prod_{j\in\Set{i..n}}{\cos_\lambda{\tensor{\theta}{^j}}}}}\\
\tensor{\theta}{^i}&=\arcsin_\lambda{\frac{\lambda\tensor{p}{^{i+1}}}{\prod_{j\in\Set{i+1..n}}{\cos_\lambda{\tensor{\theta}{^j}}}}}\\
\tensor{p}{^1}&=\frac{1}{\lambda}\prod_{j\in\Set{1..n}}{\cos_\lambda{\tensor{\theta}{^j}}}\\
\cos_\lambda{\tensor{\theta}{^1}}\prod_{j\in\Set{2..n}}{\cos_\lambda{\tensor{\theta}{^j}}}&=\lambda\tensor{p}{^1}\\
\sign\cos_\lambda{\tensor{\theta}{^1}}\prod_{j\in\Set{2..n}}\sign{\cos_\lambda{\tensor{\theta}{^j}}}&=\sign\lambda\sign\tensor{p}{^1}\\
\sign\cos_\lambda{\tensor{\theta}{^1}}\prod_{j\in\Set{2..n}}{+1}&=\sign\lambda\sign\tensor{p}{^1}\\
\sign\cos_\lambda{\tensor{\theta}{^1}}&=\sign\lambda\sign\tensor{p}{^1}\\
\theta&=\begin{pmatrix}
\arcsin_\lambda^{\sign\lambda\sign{\tensor{p}{^1}}}{\frac{\tensor{p}{^2}}{\prod_{j\in\Set{2..n}}{\cos_\lambda{\tensor{\theta}{^j}}}}}\\
\arcsin_\lambda{\frac{\tensor{p}{^{i+1}}}{\prod_{j\in\Set{i+1..n}}{\cos_\lambda{\tensor{\theta}{^j}}}}}\\
\arcsin_\lambda{\tensor{p}{^{n+1}}}\\
\end{pmatrix}
\end{align*}
\end{proof}
\begin{lemma}\label{M:CoordinateChart}
\begin{equation*}
\Psi=\Set{\psi|
\psi^{-1}
\in S^n\to P
\colon\tensor{\theta}{}\mapsto P_{\lambda,n}\left(\tensor{\theta}{}+\tensor{x}{}\right)
\text{ for }
\tensor{x}{}\in\R^n
}
\end{equation*} is a coordinate chart of a $C^\infty$ differential structure on $P$
for\begin{equation*}
S=
\begin{cases}
\left(-\frac{1}{2}\frac{\pi}{\lambda},+\frac{1}{2}\frac{\pi}{\lambda}\right)&\text{$\lambda>0$}\\
\R&\text{$\lambda\le0$}
\end{cases}
\end{equation*}
\end{lemma}
\begin{proof}[\proofof{M:CoordinateChart}]
From \cref{Manifold}, It is sufficient to shows that
\begin{APAenumerate}
\item $R_\psi$ is an open subset of real vector space (defined),
\item $\bigcup_{\psi\in\Psi} D_\psi=P$ (obvious),
\item transition map is in differentability class $C^\infty$.
\end{APAenumerate}
\begin{subproof}{$\bigcup_{\psi\in\Psi} D_\psi=P$}
\begin{align*}
M\in P
&\implies\exists\theta_0, M=P_{\lambda,n}\left(\theta_0\right)\\
&\implies\exists\theta_0, M=P_{\lambda,n}\left(0 +\theta_0\right)\\
&\implies M\in R_{\psi^{-1}}\\
&\implies M\in D_{\psi}\\
&\implies M\in\bigcup_{\psi\in\Psi} D_\psi\\
P&\subset\bigcup_{\psi\in\Psi}D_\psi\\
M\in\bigcup_{\psi\in\Psi} D_\psi
&\implies\exists\psi\in\Psi, M\in D_\psi\\
&\implies\exists\psi\in\Psi, M\in R_{\psi^{-1}}\\
&\implies\exists x_0\exists\theta\in S^n, M=P_{\lambda,n}\left(\theta+x_0\right)\\
&\implies\exists x_0, M=P_{\lambda,n}\left(0+x_0\right)\\
&\implies\exists x_0, M=P_{\lambda,n}\left(x_0\right)\\
&\implies M\in P\\
\bigcup_{\psi\in\Psi}D_\psi&\subset P\\
\bigcup_{\psi\in\Psi} D_\psi&= P
\end{align*}
\end{subproof}
\begin{subproof}{every transition map is in differentability class $C^\infty$}
Consider $\psi_1,\psi_2\in\Psi$ and $x_1, x_2\in\R^n$ where
\begin{equation*}
\psi_i^{-1}
\in S^n\to R
\colon\tensor{\theta}{}\mapsto P\left(\tensor{\theta}{}+x_i\right)
\text{.}
\end{equation*}
If $\psi_1^{-1}\left(\theta_1\right)=\psi_2^{-1}\left(\theta_2\right)$,
\begin{align*}
\psi_1^{-1}\left(\theta_1\right)
&=\psi_2^{-1}\left(\theta_2\right)\\
\psi_0^{-1}\left(\theta_1+x_1\right)
&=\psi_0^{-1}\left(\theta_2+x_2\right)\\
\psi_0^{-1}\left(\phi_1\right)
&=\psi_0^{-1}\left(\phi_2\right)&\text{(Let $\phi_i =\theta_i+x_i$)}\\
\frac{1}{\lambda}
\begin{pmatrix}
\prod_{j\in\Set{1..n}}{\cos_\lambda{\tensor{{\phi_1}}{^j}}}\\
\sin_\lambda{\tensor{{\phi_1}}{^{i-1}}}\prod_{j\in\Set{i..n}}{\cos_\lambda{\tensor{{\phi_1}}{^j}}}\\
\sin_\lambda{{\tensor{{\phi_1}}{^n}}}\\
\end{pmatrix}
&=
\frac{1}{\lambda}
\begin{pmatrix}
\prod_{j\in\Set{1..n}}{\cos_\lambda{\tensor{{\phi_2}}{^j}}}\\
\sin_\lambda{\tensor{{\phi_2}}{^{i-1}}}\prod_{j\in\Set{i..n}}{\cos_\lambda{\tensor{{\phi_2}}{^j}}}\\
\sin_\lambda{{\tensor{{\phi_2}}{^n}}}\\
\end{pmatrix}&\text{\cref{M:Vector:Value}}\\
\begin{pmatrix}
\prod_{j\in\Set{1..n}}{\cos_\lambda{\tensor{{\phi_1}}{^j}}}\\
\sin_\lambda{\tensor{{\phi_1}}{^{i-1}}}\prod_{j\in\Set{i..n}}{\cos_\lambda{\tensor{{\phi_1}}{^j}}}\\
\sin_\lambda{{\tensor{{\phi_1}}{^n}}}\\
\end{pmatrix}
&=
\begin{pmatrix}
\prod_{j\in\Set{1..n}}{\cos_\lambda{\tensor{{\phi_2}}{^j}}}\\
\sin_\lambda{\tensor{{\phi_2}}{^{i-1}}}\prod_{j\in\Set{i..n}}{\cos_\lambda{\tensor{{\phi_2}}{^j}}}\\
\sin_\lambda{{\tensor{{\phi_2}}{^n}}}\\
\end{pmatrix}
\end{align*}
If $\lambda\ge 0$, by mathematical induction, $\phi_1=m\frac{\pi}{\lambda}\pm\phi_2$.
If $\lambda\le 0$, by mathematical induction, $\phi_1 =\phi_2$.
Hence, the transition map $\tau_{1,2} =\psi_2\circ\psi_1^{-1}$
is in the form of linear function $\theta\mapsto c\pm\theta$
and is in differentability class $C^\infty$.
\end{subproof}
\end{proof}
\subsection{Locus of position vector}
\begin{lemma}\label{SphericalLocus}
For $\lambda>0$, $P$ is a $\left(n+1\right)$-sphere of radius $\abs{\lambda}^{-1}$.
\end{lemma}
\begin{proof}[\proofof{SphericalLocus}]
Simplify \cref{M:Vector:Value} using \cref{M:Trigonometry:Pythagorean}.
\end{proof}
\begin{lemma}\label{HyperbolicLocus}
For $\lambda<0$, $P$ is a backward sheet of a two-sheeted $\left(n+1\right)$-hyperboloid of radius $\abs{\lambda}^{-1}$.
\end{lemma}
\begin{proof}[\proofof{HyperbolicLocus}]
Simplify \cref{M:Vector:Value} using \cref{M:Trigonometry:Pythagorean}.
\end{proof}
\begin{lemma}\label{EuclideanLocus}
For $\lambda\to0$, $P$ is a $n$-Euclidean plane at $\tensor{p}{^1}=\infty$.
\end{lemma}
\begin{proof}[\proofof{EuclideanLocus}]
Using limits.
\end{proof}
\end{document}