% !TeX root = ../main.tex
\documentclass[../main.tex]{subfiles}
\begin{document}
\section{Preliminary}
\subsection{Tensor}
\begin{equation}\label{Tensor:Sum}
\tensor{\left(A+B\right)}{
^{i_1}^{\dots}^{i_n}
_{j_1}_{\dots}_{j_m}
}
=
\tensor{A}{
^{i_1}^{\dots}^{i_n}
_{j_1}_{\dots}_{j_m}
}
+
\tensor{B}{
^{i_1}^{\dots}^{i_n}
_{j_1}_{\dots}_{j_m}
}
\end{equation}
\begin{equation}\label{Tensor:Product:Scalar}
\tensor{\left(\alpha A\right)}{
^{i_1}^{\dots}^{i_n}
_{j_1}_{\dots}_{j_m}
}
=
\alpha
\tensor{A}{
^{i_1}^{\dots}^{i_n}
_{j_1}_{\dots}_{j_m}
}
\end{equation}
\begin{equation}\label{Tensor:Product:Tensor}
\tensor{\left(A\otimes B\right)}{
^{i_1}^{\dots}^{i_l}^{i_{l+1}}^{\dots}^{i_{l+n}}
_{j_1}_{\dots}_{j_k}_{j_{k+1}}_{\dots}_{j_{k+m}}
}
=
\tensor{A}{
^{i_1}^{\dots}^{i_l}
_{j_1}_{\dots}_{j_k}
}
\tensor{B}{
^{i_{l+1}}^{\dots}^{i_{l+n}}
_{j_{k+1}}_{\dots}_{j_{k+m}}
}
\end{equation}
\begin{equation}\label{Tensor:Contraction}
\tensor{\left(\contr{\tensor{T}{}}\right)}{
^{i_1}^{\dots}^{i_n}
_{j_1}_{\dots}_{j_m}
}
=
\sum_a{\tensor{T}{
^{i_1}^{\dots}^{i_n}^a
_a_{j_1}_{\dots}_{j_m}
}}
\end{equation}
\begin{equation}\label{Matrix:Product}
\tensor{\left(AB\right)}{^i_j}
=\contr{\left(A\otimes B\right)}
=\sum_k\tensor{A}{^i_k}\tensor{B}{^k_j}
\end{equation}
\begin{equation}\label{Matrix:Product:Block}
\left(\begin{bmatrix}
A_{11}&A_{12} \\
A_{21}&A_{22} \\
\end{bmatrix}
\begin{bmatrix}
B_{11}&B_{12} \\
B_{21}&B_{22} \\
\end{bmatrix}\right)_{ij}
=
\sum_k A_{ik} B_{kj}
\end{equation}
\begin{equation}\label{Matrix:Identity}
IA = A = AI
\end{equation}
\begin{equation}\label{Matrix:Identity:Value}
I_n=\diag{\left(1,1,\dots,1\right)}
\end{equation}
\begin{equation}\label{Matrix:Identity:Block}
I_{a+b}=
\begin{bmatrix}
I_a&0_{a\times b} \\
0_{b\times a}&I_b           \\
\end{bmatrix}
\end{equation}
\begin{equation}\label{Matrix:Permutation:Square}
T_{a,b} T_{a,b}= I
\end{equation}
\subsection{Differential}
\begin{equation}
J_xF = \left(\left.\frac{\partial F_i}{\partial x_j}\right|_x\right)_{i,j}
\end{equation}
\begin{equation}
\left.Df\right|_x
: T_x\R^k\to T_{f(x)}\R^n
= \left(x,v\right) \mapsto \left(f(x),J_xf(v)\right)
\end{equation}
\subsection{Group}
\begin{definition}[Lie group {\autocite[][Chapter~7]{lee_2013}}]\label{Group:Lie}
A \textit{Lie group} is a smooth manifold \(G\) (without boundary)
that is also a group in the algebraic sense,
with the property that
the multiplication map \(m\colon G\times G\to G\)
and the inversion map \(i\colon G\to G\), given by
\[
m\left(g,h\right)=gh\text{,}
\quad\quad
i\left(g\right)=g^{-1}
\]
are both smooth.
\end{definition}
\begin{proposition}[Lie group {\autocite[][Chapter~7]{lee_2013}}]\label{Group:Lie:Assertion}
If \(G\) is a smooth manifold with a group structure such that the map \(G\times G\to G\) given by \(\left(g,h\right)\mapsto gh^{-1}\) is smooth, then \(G\) is a Lie group.
\end{proposition}
\begin{definition}[Semidirect product]\label{Group:SemidirectProduct}
Suppose \(H\) and \(N\) are groups,
and \(\theta\colon H\times N\to N\) is a smooth left action of \(H\) on \(N\).
It is said to be an \textit{action by automorphisms}
if for each \(h\in H\), the map \(\theta_h\colon N\to N\) is a group automorphism of \(N\) (i.e., an isomorphism from \(N\) to itself).
Given such action, we define a new group \(N\rtimes_\theta H\),
called a \textit{semidirect product} of \(H\) and \(N\), as follows.
\(N\rtimes_\theta H\) is just the Cartesian product \(N\times H\);
but the group multiplication is defined by
\[\left(n,h\right)\left(n^\prime,h^\prime\right)=\left(n\theta_h\left(n^\prime\right),hh^\prime\right)\text{.}\]
\end{definition}
\begin{definition}\label{OrthogonalGroup}
An orthogonal group \(O\left(n\right)\) is a group of orthogonal matrix \(M\) where \(M^{-1}=M^T\) with matrix multiplication.
\end{definition}
\begin{definition}\label{SpecialOrthogonalGroup}
An special orthogonal group \(SO\left(n\right)\) is a subgroup of the orthogonal group \(O\left(n\right)\) whose element \(M\) have the property that \(\det{M}=1\).
\end{definition}
\begin{definition}\label{IndefiniteOrthogonalGroup}
An indefinite orthogonal group \(O\left(m,n\right)\) is a group of indefinite orthogonal matrix \(M\) where \(M^{-1}=gM^Tg\) for \(g=\diag{\left(-1,\dots,-1,1,\dots,1\right)}\) with matrix multiplication.
\end{definition}
\begin{definition}\label{OrthochronusIndefiniteOrthogonalGroup}
An orthochronus indefinite orthogonal group \(O^{+}\left(m,n\right)\) is a subgroup of the indefinite orthogonal group \(O\left(m,n\right)\) whose element \(M\) have the property that \(\tensor{M}{^1_1}>0\).
\end{definition}
\begin{definition}\label{SpecialOrthochronusIndefiniteOrthogonalGroup}
An special orthochronus indefinite orthogonal group \(SO^{+}\left(m,n\right)\) is a subgroup of the orthochronus indefinite orthogonal group \(O^{+}\left(m,n\right)\) whose element \(M\) have the property that \(\det{M}=1\).
\end{definition}
\begin{definition}\label{TranslationGroup}
An translation group \(T\left(n\right)\) is a group of vectors in \(\R^n\) with vector addition.
\end{definition}
\begin{definition}\label{EuclideanGroup}
An euclidean group \(E\left(n\right)\) is a semidirect product of orthogonal group \(O\left(n\right)\) extended by translation group \(T\left(n\right)\).
\end{definition}
\subsection{Manifold}
\begin{definition}[Abstract differentiable manifold {\autocite[][Chapter~5A]{kuhnelwolfgang_2006}}]\label{Manifold}
A \textit{\(k\)-dimensional differentiable manifold} (briefly: a \(k\)-manifold)
is a set \(M\) together with a family \(\left(M_i\right)_{i\in I}\) of subsets such that
\begin{APAenumerate}
\item \(M=\bigcup_{i\in I} M_i\) (union),
\item for every \(i\in I\) there is an injective map \(\varphi_i\colon M_i\to\R^k\) so that \(\phi_i\left(M_i\right)\) is open in \(\R^k\), and
\item for \(M_i\cap M_j\ne\emptyset\), \(\varphi_i\left(M_i\cap M_j\right)\) is open in \(\R^k\).
\end{APAenumerate}
\end{definition}
\begin{definition}[Structures on a manifold {\autocite[][Chapter~5A]{kuhnelwolfgang_2006}}]\label{Manifold:Extended}
Given a \(k\)-dimensional differentiable manifold,
one gets additional structure
by replacing aditional requirements on the transformation functions \(\varphi_j\circ\varphi_i^{-1}\),
which belong to the atlas of the manifold;
if all \(\varphi_j\circ\varphi_i^{-1}\) are (left-hand side),
then one speaks of (right-hand side) as follows:
\begin{center}
\begin{tabular}{ r c l }
continuous&\(\leftrightarrow\)&topological manifold                                 \\
differentiable&\(\leftrightarrow\)&differentiable manifold                              \\
\(C^1\)-differentiable&\(\leftrightarrow\)&\(C^1\)-manifold                                       \\
\(C^r\)-differentiable&\(\leftrightarrow\)&\(C^r\)-manifold                                       \\
\(C^\infty\)-differentiable&\(\leftrightarrow\)&\(C^\infty\)-manifold                                  \\
real analytic&\(\leftrightarrow\)&real analytic manifold                               \\
complex analytic&\(\leftrightarrow\)&complex analytic manifold of dimension \(\frac{k}{2}\) \\
affine&\(\leftrightarrow\)&affine manifold                                      \\
projective&\(\leftrightarrow\)&projective manifold                                  \\
conformal&\(\leftrightarrow\)&manifold with a conformal structure                  \\
orienatation-preserving&\(\leftrightarrow\)&orientable manifold                                  \\
\end{tabular}
\end{center}
\end{definition}
\begin{definition}[Tangent vector {\autocite[][Chapter~5B]{kuhnelwolfgang_2006}}]\label{Manifold:TangentVector}
A \textit{tangent vector} \(X\) at \(p\)
is a derivation (derivative operator) defined on the set of \textit{germs of functions}
\[\mathcal{F}_p\left(M\right)\coloneqq\Set{f\colon M\to\R|f\text{ differentiable}}/\sim\text{,}\]
where the equivalence relation \(\sim\) is defined by
declaring \(f\sim f^\ast\) if and only if
\(f\) and \(f^\ast\) coincide in a neighborhood of \(p\).
The value \(X\left(f\right)\) is also referred to as the \textit{directional derivative} of \(f\) in the direction \(X\).
This definition means more precisely the following.
\(X\) is a map \(X\colon\mathcal{F}_p\left(M\right)\to\R\)
with the two following properties:
\begin{APAenumerate}
\item \(X\left(\alpha f+\beta g\right)=\alpha X\left(f\right)+\beta\left(g\right)\), \(f,g\in\mathcal{F}_p\left(M\right)\) (\textit{\(\R\)-linearity});
\item \(X\left(f\cdot g\right)=X\left(f\right)\cdot g\left(p\right)+f\left(p\right)\cdot X\left(g\right)\) for \(f,g\in\mathcal{F}_p\left(M\right)\) (\textit{product rule}).
\end{APAenumerate}
(For this to make sense, both \(f\) and \(g\) have to be defined in a neighborhood of \(p\).)
Briefly: \textit{tangent vectors are derivations acting on scalar functions.}
\end{definition}
\begin{corollary}[Tangent space {\autocite[][Chapter~5B]{kuhnelwolfgang_2006}}]\label{Manifold:TangentSpace}
The \textit{tangent space} \(T_pM\) of \(M\) at \(p\)
is defined in all cases as
the set of all tangent vectors at the point \(p\).
By definition \(T_pM\) and \(T_qM\) are disjoint if \(p\ne q\).
\end{corollary}
\begin{definition}[Riemannian metric {\autocite[][Chapter~5C]{kuhnelwolfgang_2006}}]\label{Manifold:RiemannianMetric}
A \textit{Riemannian metric} \(g\) on \(M\)
is an association \(p\mapsto g_p\in L^2\left(T_pM;\R\right)\)
such that the following conditins are satisfied:
\begin{APAenumerate}
\item \(g_p\left(X,Y\right)=g_p\left(Y,X\right)\) for all \(X\), \(Y\), \hfill (\textit{symmetry})
\item \(g_p\left(X,X\right)>0\) for all \(X\ne0\), \hfill (\textit{positive definiteness})
\item The coefficient \(\tensor{g}{_i_j}\) in every local representation (i.e., in every chart) \[g_p=\sum_{i,j}\tensor{g}{_i_j}\left(p\right)\cdot \left.d\tensor{x}{^i}\right|_p \otimes\left.d\tensor{x}{^j}\right|_p\] are differentiable functions. \hfill (\textit{differentiability})
\end{APAenumerate}
\end{definition}
\begin{remark}\label{Manifold:Riemannian}
The pair \(\left(M,g\right)\) is then called \textit{Riemannian manifold}.
One also refers to the Riemannian metric as the \textit{metric tensor}.
In local coordinates the metric tensor is given by the matrix (\(\tensor{g}{_i_j}\)) of functions.
In Ricci calculus this is simply written as \(\tensor{g}{_i_j}\).
{\autocite[][Chapter~5C]{kuhnelwolfgang_2006}}
\end{remark}
\begin{remark}\label{Manifold:InnerProduct}
A Riemannian metric \(g\) defines at every point \(p\)
an \textit{inner product} \(g_p\) on the tangent space \(T_pM\),
and therefore the notation \(\inner{X}{Y}\) instead of \(g_p\left(X,Y\right)\) is also used.
The notions of angles and lengths are determined by this inner product,
just as these notions are determined by the first fundamental form on surface elements.
The length or norm of vector \(X\) is given by \(\norm{X}\coloneqq\sqrt{g\left(X,X\right)}\),
and the angle \(\beta\) between two tangent vectors \(X\) and \(Y\)
can be defined by the validity of the equation \(\cos\beta\cdot\norm{X}\cdot\norm{Y}=g\left(X,Y\right)\).
{\autocite[][Chapter~5C]{kuhnelwolfgang_2006}}
\end{remark}
\subsection{Curvature}
\begin{definition}[The Lie bracket {\autocite[][Chapter~5D]{kuhnelwolfgang_2006}}]\label{Manifold:LieBracket}
Let \(X\), \(Y\) be (differentiable) vector fields on \(M\),
and let \(f\colon M\to\R\) be a differentiable function.
Through the relation
\[
\left[X,Y\right]\left(f\right)
\coloneqq
X\left(Y\left(f\right)\right)
- Y\left(X\left(f\right)\right)
\]
we define a vector field \(\left[X,Y\right]\),
which is referred to as the \textit{Lie bracket} of \(X\), \(Y\)
(also called the \textit{Lie derivative} \(\mathcal{L}_XY\) of \(Y\) in the direction \(X\)).
At a point \(p\in M\) we have \(\left[X,Y\right]_p\left(f\right)=X_p\left(Yf\right)-Y_p\left(Xf\right)\).
\end{definition}
\begin{lemma}[Properties of the Lie bracket {\autocite[][Chapter~5D]{kuhnelwolfgang_2006}}]\label{Manifold:LieBracket:Property}
Let \(X\), \(Y\), \(Z\) be vector fields,
let \(\alpha\), \(\beta\) be real constants,
and let \(f, h\colon M\to\R\) be differentiable functions.
Then the Lie bracket has the following properties:
\begin{APAenumerate}
\item \(\left[\alpha X+\beta Y, Z\right]=\alpha\left[X,Z\right]+\beta\left[Y,Z\right]\);
\item \(\left[X,Y\right]=-\left[Y,X\right]\);
\item \(\left[fX,hY\right]=f\cdot h\cdot\left[X,Y\right]+f\cdot\left(Xh\right)\cdot Y-h\cdot\left(Yf\right)\cdot X\);
\item \(\left[X,\left[Y,Z\right]\right]+\left[Y,\left[Z,X\right]\right]+\left[Z,\left[X,Y\right]\right] = 0\); (\textit{Jacobi identity})
\item \(\left[\frac{\partial}{\partial\tensor{x}{^i}}, \frac{\partial}{\partial\tensor{x}{^j}}\right]=0\) for every chart with coordinates (\(\tensor{x}{^1},\dots,\tensor{x}{^n}\));
\item \(\left[\sum_i\tensor{\xi}{^i}\frac{\partial}{\partial\tensor{x}{^i}},\sum_j\tensor{\eta}{^j}\frac{\partial}{\partial\tensor{x}{^j}}\right]=\sum_{i,j}\left(\tensor{\xi}{^i}\frac{\partial\tensor{\eta}{^j}}{\partial\tensor{x}{^i}}-\tensor{\eta}{^i}\frac{\partial\tensor{\xi}{^j}}{\partial\tensor{x}{^i}}\right)\frac{\partial}{\partial\tensor{x}{^j}}\) (\textit{representation in local coordinates}).
\end{APAenumerate}
\end{lemma}
\begin{definition}[Riemannian connection {\autocite[][Chapter~5D]{kuhnelwolfgang_2006}}]\label{Manifold:Connection}
A \textit{Riemannian connection} \(\nabla\) (pronounced "nabla")
on a Riemannian manifold \(\left(M,g\right)\) is a map
\[\left(X,Y\right)\mapsto\nabla_XY\text{,}\]
which associates to two given differentiable vector fields \(X\), \(Y\)
a third differentiable vector field \(\nabla_XY\),
such that the following conditions are satisfied: (\(f\colon M\to\R\) denotes a differentiable function):
\begin{APAenumerate}
\item \(\nabla_{X_1+X_2}Y=\nabla_{X_1}Y+\nabla_{X_2}Y\); (\textit{additivity in the subscript})
\item \(\nabla_{fX}Y=f\cdot\nabla_{X}Y\); (\textit{linearity in the subscript})
\item \(\nabla_X\left(Y_1+Y_2\right)=\nabla_XY_1+\nabla_XY_2\); (\textit{additivity in the argument})
\item \(\nabla_X\left(fY\right)=f\cdot\nabla_XY+\left(X\left(f\right)\right)\cdot Y\); (\textit{product rule in the argument})
\item \(X\left(g\left(Y,Z\right)\right)=g\left(\nabla_XY,Z\right)+g\left(Y,\nabla_XZ\right)\); (\textit{compatibility with the metric})
\item \(\nabla_XY-\nabla_YX-\left[X,Y\right]=0\). (\textit{symmetry or torsion-freeness})
\end{APAenumerate}
\end{definition}
\begin{remark}
We get the following expression for \(\nabla_XY\), in local coordinates, provided \(X=\sum_i\tensor{\xi}{^i}\frac{\partial}{\partial\tensor{x}{^i}}\) and \(Y=\sum_j\tensor{\eta}{^j}\frac{\partial}{\partial\tensor{x}{^j}}\):
\[
\nabla_XY=\sum_k\left(\sum_i\tensor{\xi}{^i}\frac{\partial\tensor{\eta}{^k}}{\partial\tensor{x}{^i}}+\sum_{i,j}\tensor{\Gamma}{^k_i_j}\tensor{\xi}{^i}\tensor{\eta}{^j}\right)\frac{\partial}{\partial\tensor{x}{^k}}\text{.}
\]
Especially for \(X=\frac{\partial}{\partial\tensor{x}{^i}}\) we obtain
\[
\nabla_XY=\sum_k\left(\frac{\partial\tensor{\eta}{^k}}{\partial\tensor{x}{^i}}+\sum_j\tensor{\Gamma}{^k_i_j}\tensor{\eta}{^j}\right)\frac{\partial}{\partial\tensor{x}{^k}}\text{.}
\]
Consequently, in Ricci calculus the notation for this formular is
\[
\nabla_i\tensor{\eta}{^k}=\frac{\partial\tensor{\eta}{^k}}{\partial\tensor{x}{^i}}+\tensor{\Gamma}{^k_i_j}\tensor{\eta}{^j}\text{.}
\]
\end{remark}
\begin{definition}[Christoffel symbols {\autocite[][Chapter~4A]{kuhnelwolfgang_2006}}]\label{Manifold:ChristoffelSymbol}
\;\newline
\begin{APAenumerate}
\item The quantities \(\partial_k\tensor{\Gamma}{_i_j}\) defined by the expressions
\[
\partial_k\tensor{\Gamma}{_i_j}
\coloneqq
\inner{\nabla_{\frac{\partial f}{\partial\tensor{u}{^i}}}\frac{\partial f}{\partial\tensor{u}{^j}}}{\frac{\partial f}{\partial\tensor{u}{^k}}}
\]
are called the \textit{Christoffel symbols of the first kind}.
\item The quantities \(\tensor{\Gamma}{^k_i_j}\) defined by the expressions
\[
\nabla_{\frac{\partial f}{\partial\tensor{u}{^i}}}\frac{\partial f}{\partial\tensor{u}{^j}}
=
\sum_k{\tensor{\Gamma}{^k_i_j}\frac{\partial f}{\partial\tensor{u}{^k}}}
\]
are called the \textit{Christoffel symbols of the second kind}.
\item By definition one has \(\partial_k\tensor{\Gamma}{_i_j}=\partial_k\tensor{\Gamma}{_j_i}\), \(\tensor{\Gamma}{^k_i_j}=\tensor{\Gamma}{^k_j_i}\)
as well as \(\partial_k\tensor{\Gamma}{_i_j}=\sum_{m}\tensor{\Gamma}{^m_i_j}\tensor{g}{_m_k}\).
\end{APAenumerate}
\end{definition}
\begin{definition}[Curvature tensor {\autocite[][Chapter~4C]{kuhnelwolfgang_2006}}]\label{Manifold:CurvatureTensor}
\[
R\left(X,Y\right)Z
\coloneqq
\nabla_X\nabla_YZ - \nabla_Y\nabla_XZ - \nabla_{\lie{X}{Y}}Z
\]
is a tensor field, which is called the \textit{curvature tensor} of the surface.
\end{definition}
\begin{remark}[Curvature tensor {\autocite[][Chapter~4C]{kuhnelwolfgang_2006}}]\label{Manifold:CurvatureTensor:ChristoffelSymbol}
The left-hand side of the Gauss equation is called the \textit{curvature tensor}
and is in general expressed in the form
\[
\tensor{R}{^s_i_k_j}
\coloneqq
\frac{\partial}{\partial\tensor{u}{^k}}\tensor{\Gamma}{^s_i_j}
- \frac{\partial}{\partial\tensor{u}{^j}}\tensor{\Gamma}{^s_i_k}
+ \sum_{r}\left(\tensor{\Gamma}{^r_i_j}\tensor{\Gamma}{^s_r_k}-\tensor{\Gamma}{^r_i_k}\tensor{\Gamma}{^s_r_j}\right) \text{.}
\]
\end{remark}
\begin{definition}[ {\autocite[][Chapter~6B]{kuhnelwolfgang_2006}}]
With respect to a given Riemannian metric \(\inner{}{}\),
the \textit{standard curvature tensor} \(\tensor{{R_1}}{}\) is defined by the relation
\(R_1\left(X,Y\right)Z\coloneqq\inner{Y}{Z}X-\inner{X}{Z}Y\).
We then set
\begin{align*}
\kappa_1\left(X,Y\right)
\coloneqq\inner{R_1\left(X,Y\right)Y}{X}
=\inner{X}{X}\inner{Y}{Y}-\inner{X}{Y}^2\text{,}\\
\kappa\left(X,Y\right)
\coloneqq\inner{R\left(X,Y\right)Y}{X}\text{.}
\end{align*}
Let \(\sigma\subset T_pM\) be a two-dimensional subspace, spanned by \(X\), \(Y\).
Then the quantity
\[
K_\sigma
\coloneqq
\frac{\kappa\left(X,Y\right)}{\kappa_1\left(X,Y\right)}
\]
is called the \textit{sectional curvature} of the Riemannian manifold with respect to the plane \(\sigma\).
\end{definition}
\begin{definition}[Space of constant curvature {\autocite[][Chapter~6B]{kuhnelwolfgang_2006}}]\label{Manifold:Spaceform}
If on a Riemannian manifold \(K_\sigma\) is constant
or, equalently,
if \(R=K\cdot R_1\) with \(K\in\R\),
the manifold is called a \textit{space of constant curvature}.
\end{definition}
\begin{remark}
By \textit{scaling} one means the process of
replacing a metric \(g\) by \(\tilde{g}]\coloneqq\lambda^2g\), where \(\lambda\ne0\) is a constant.
In this case one has \(\tilde{R_1}\left(X,Y\right)Z=\lambda^2R_1\left(X,Y\right)Z\).
On the other hand we have
\(\tilde{\tensor{\Gamma}{^k_i_j}}=\tensor{\Gamma}{^k_i_j}\),
\(\tilde{\nabla}_XY=\nabla_XY\) and, consequently,
\(\tilde{R}\left(X,Y\right)Z=R\left(X,Y\right)Z\)
as well as \(\tilde{K}=K\lambda^{-2}\).
Hence in \cref{Manifold:Spaceform} there are (up to scaling) only three possible curvature tensor with constant curvature:
\begin{align*}
R=R_1&\text{(with \(K=1\)),}\\
R=0&\text{(with \(K=0\)),}\\
R=R_{-1}\coloneqq-R_1&\text{(with \(K=-1\)).}
\end{align*}
Model spaces for these are the sphere \(S^n\), the euclidean space \(E^n\) and the hyperbolic space \(H^n\)
\end{remark}
\subsection{Space of Constant Curvature}
\begin{definition}(Pseudo-Euclidean space \(\R^n_k\) {\autocite[][Chapter~7A]{kuhnelwolfgang_2006}})
The so-called \textit{pseudo-Euclidean} metric (or pseudo-Euclidean inner product)
\[
g\left(X,X\right)
=\inner{X}{X}_k
=-\sum_{i=1}^{k}x_i^2+\sum_{i=k+1}^{n}x_i^2
\]
for a vector \(X\) with components \(s_1,\dots,x_n\),
where \(0\le k\le n\) is a fixed number which is called the \textit{index} or the \textit{signature} of the inner produce.
The pair \(\left(\R^n,g\right)\) is then called a \textit{pesudo-Euclidean space}
and is denoted by \(\R^n_k\) or \(E^n_k\).
In particular, \(E^n=R^n_0\) is the usual Euclidean space.
\end{definition}
\begin{definition}(The sphere \(S^n\) {\autocite[][Chapter~7A]{kuhnelwolfgang_2006}})
The sphere with its spherical metric is most easily defined as a hypersurface in Euclidean space with the associated first fundamental form, that is,
\[
S^n\coloneqq\Set{X\in\R^{n+1}|\inner{X}{X}=\sum_ix_i^2=1}\text{.}
\]
The Gauss equation then implies the sectional curvature to be \(K_\sigma=+1\) at every point \(p\) and every plane \(\sigma\subseteq T_pS^n\).
\end{definition}
\begin{definition}(Hyperbolic space \(H^n\) {\autocite[][Chapter~7A]{kuhnelwolfgang_2006}})
We defined the \(n\)-dimensional \textit{hyperbolic space} \(H^n\)
as the component of \(\Set{X\in\R^{n+1}_1|\inner{X}{X}_1=-1}\)
which contains the point \(\left(+1,0,\dots,0\right)\),
that is, the upper component of the two-sheeted hyperboloid.
The sectional curvature of hyperbolic space defined in this manner is constant: \(K=-1\).
\end{definition}
\begin{definition}(Symmetries of the space \(E^n\) {\autocite[][Chapter~7A]{kuhnelwolfgang_2006}})\label{Group:Euclidean}
The group \(E\left(n\right)\) of Euclidean motions
(the so-called \textit{Euclidean group})
acts on \(E^n\).
This group contains in particular all transpations
(these form a subgroup which is isomorphic to \(\R^n\), in fact a normal subgroup of \(E\left(n\right)\))
as well as the rotation group \(O\left(n\right)\),
consisting of symmetries which leave a point invariant.
In fact, \(E\left(n\right)\) is a semi-direct product of these two subgroups.
\end{definition}
\begin{definition}(Symmetries of the space \(S^n\) {\autocite[][Chapter~7A]{kuhnelwolfgang_2006}})\label{Group:Sphere}
The \textit{orthogonal group}
\[
O\left(n+1\right)
=
\Set{A\colon\R^{n+1}\to\R^{n+1}|\text{\(A\) preserves the Euclidean inner product}}
\]
acts on the sphere \(S^n\).
Here, \(A\) denotes a linear map.
As is well-known, \(A\in O\left(n+1\right)\) holds if and only if \(A^T=A^{-1}\).
As a matter of fact, the orthogonal group acts on the entire space \(\R^{n+1}\),
but we can consider its action when restricted to the sphere and denote the group in the same way.
\end{definition}
\begin{definition}(Symmetries of the space \(H^n\) {\autocite[][Chapter~7A]{kuhnelwolfgang_2006}})\label{Group:Hyperbolic}
The \textit{Lorentz group}
\[
O\left(n,1\right)
=
\Set{A\colon\R^{n+1}_1\to\R^{n+1}_1|\text{\(A\) preserves the pseudo-Euclidean inner product}}
\]
acts on \textit{Lorentz space} or on \textit{Minkowski space} \(\R^{n+1}_1\)
and preserves the set \(\tilde{H}=\Set{X|\inner{X}{X}_1=-1}\).
The "positive" part of this set,
\[
O_{+}\left(n,1\right)
=
\Set{A\colon O\left(n,1\right)|\text{\(A\) preserves \(\tilde{H}\cap\Set{x_0>0}\)}}
\]
then acts on hyperbolic space \(H^n\) and preserves its metric.
\end{definition}
\end{document}