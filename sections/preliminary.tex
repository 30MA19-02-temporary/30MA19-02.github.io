% !TeX root = ../main.tex

\documentclass[../main.tex]{subfiles}

\begin{document}

\section{Preliminary}

\subsection{Tensor}

\begin{equation}\label{Tensor:Sum}
    \tensor{\left(A+B\right)}{
        ^{i_1}^{\dots}^{i_n}
            _{j_1}_{\dots}_{j_m}
    }
    =
    \tensor{A}{
        ^{i_1}^{\dots}^{i_n}
            _{j_1}_{\dots}_{j_m}
    }
    +
    \tensor{B}{
        ^{i_1}^{\dots}^{i_n}
            _{j_1}_{\dots}_{j_m}
    }
\end{equation}
\begin{equation}\label{Tensor:Product:Scalar}
    \tensor{\left(\alpha A\right)}{
        ^{i_1}^{\dots}^{i_n}
            _{j_1}_{\dots}_{j_m}
    }
    =
    \alpha
    \tensor{A}{
        ^{i_1}^{\dots}^{i_n}
            _{j_1}_{\dots}_{j_m}
    }
\end{equation}
\begin{equation}\label{Tensor:Product:Tensor}
    \tensor{\left(A\otimes B\right)}{
        ^{i_1}^{\dots}^{i_l}^{i_{l+1}}^{\dots}^{i_{l+n}}
            _{j_1}_{\dots}_{j_k}_{j_{k+1}}_{\dots}_{j_{k+m}}
    }
    =
    \tensor{A}{
        ^{i_1}^{\dots}^{i_l}
            _{j_1}_{\dots}_{j_k}
    }
    \tensor{B}{
        ^{i_{l+1}}^{\dots}^{i_{l+n}}
            _{j_{k+1}}_{\dots}_{j_{k+m}}
    }
\end{equation}
\begin{equation}\label{Tensor:Contraction}
    \tensor{\left(\contr{\tensor{T}{}}\right)}{
        ^{i_1}^{\dots}^{i_n}
            _{j_1}_{\dots}_{j_m}
    }
    =
    \sum_a{\tensor{T}{
    ^{i_1}^{\dots}^{i_n}^a
    _a_{j_1}_{\dots}_{j_m}
    }}
\end{equation}
\begin{equation}\label{Matrix:Product}
    \tensor{\left(AB\right)}{^i_j}
    =\contr{\left(A\otimes B\right)}
    =\sum_k\tensor{A}{^i_k}\tensor{B}{^k_j}
\end{equation}
\begin{equation}\label{Matrix:Product:Block}
    \left(\begin{bmatrix}
        A_{11} & A_{12} \\
        A_{21} & A_{22} \\
    \end{bmatrix}
    \begin{bmatrix}
        B_{11} & B_{12} \\
        B_{21} & B_{22} \\
    \end{bmatrix}\right)_{ij}
    =
    \sum_k A_{ik} B_{kj}
\end{equation}
\begin{equation}\label{Matrix:Identity}
    IA = A = AI
\end{equation}
\begin{equation}\label{Matrix:Identity:Value}
    I_n=\diag{\left(1,1,\dots,1\right)}
\end{equation}
\begin{equation}\label{Matrix:Identity:Block}
    I_{a+b}=
    \begin{bmatrix}
        I_a           & 0_{a\times b} \\
        0_{b\times a} & I_b           \\
    \end{bmatrix}
\end{equation}
\begin{equation}\label{Matrix:Permutation:Square}
    T_{a,b} T_{a,b}= I
\end{equation}

\subsection{Differential}
\begin{equation}
    J_xF = \left(\left.\frac{\partial F_i}{\partial x_j}\right|_x\right)_{i,j}
\end{equation}
\begin{equation}
    \left.Df\right|_x
    : T_x\R^k\to T_{f(x)}\R^n
    = \left(x,v\right) \mapsto \left(f(x),J_xf(v)\right)
\end{equation}

\subsection{Group}

\begin{definition}[Lie group {\autocite[][Chapter~7]{lee_2013}}]\label{Group:Lie}
    A \textit{Lie group} is a smooth manifold $G$ (without boundary)
    that is also a group in the algebraic sense,
    with the property that
    the multiplication map $m:G\times G\to G$
    and the inversion map $i:G\to G$, given by
    \begin{equation*}
        m\left(g,h\right)=gh\text{,}
        \quad\quad
        i\left(g\right)=g^{-1}
    \end{equation*}
    are both smooth.
\end{definition}

\begin{proposition}[Lie group {\autocite[][Chapter~7]{lee_2013}}]\label{Group:Lie:Assertion}
    If $G$ is a smooth manifold with a group structure such that the map $G\times G\to G$ given by $\left(g,h\right)\mapsto gh^{-1}$ is smooth, then $G$ is a Lie group.
\end{proposition}

\begin{definition}[Semidirect product]\label{Group:SemidirectProduct}
    Suppose $H$ and $N$ are groups,
    and $\theta:H\times N\to N$ is a smooth left action of $H$ on $N$.
    It is said to be an \textit{action by automorphisms}
    if for each $h\in H$, the map $\theta_h:N\to N$ is a group automorphism of $N$ (i.e., an isomorphism from $N$ to itself).
    Given such action, we define a new group $N\rtimes_\theta H$,
    called a \textit{semidirect product} of $H$ and $N$, as follows.
    $N\rtimes_\theta H$ is just the Cartesian product $N\times H$;
    but the group multiplication is defined by
    $$\left(n,h\right)\left(n^\prime,h^\prime\right)=\left(n\theta_h\left(n^\prime\right),hh^\prime\right)\text{.}$$
\end{definition}

\begin{definition}\label{OrthogonalGroup}
    An orthogonal group $O\left(n\right)$ is a group of orthogonal matrix $M$ where $M^{-1}=M^T$ with matrix multiplication.
\end{definition}
\begin{definition}\label{SpecialOrthogonalGroup}
    An special orthogonal group $SO\left(n\right)$ is a subgroup of the orthogonal group $O\left(n\right)$ whose element $M$ have the property that $\det{M}=1$.
\end{definition}
\begin{definition}\label{IndefiniteOrthogonalGroup}
    An indefinite orthogonal group $O\left(m,n\right)$ is a group of indefinite orthogonal matrix $M$ where $M^{-1}=gM^Tg$ for $g=\diag{\left(-1,\dots,-1,1,\dots,1\right)}$ with matrix multiplication.
\end{definition}
\begin{definition}\label{OrthochronusIndefiniteOrthogonalGroup}
    An orthochronus indefinite orthogonal group $O^+\left(m,n\right)$ is a subgroup of the indefinite orthogonal group $O\left(m,n\right)$ whose element $M$ have the property that $\tensor{M}{^1_1}>0$.
\end{definition}
\begin{definition}\label{SpecialOrthochronusIndefiniteOrthogonalGroup}
    An special orthochronus indefinite orthogonal group $SO^+\left(m,n\right)$ is a subgroup of the orthochronus indefinite orthogonal group $O^+\left(m,n\right)$ whose element $M$ have the property that $\det{M}=1$.
\end{definition}
\begin{definition}\label{TranslationGroup}
    An translation group $T\left(n\right)$ is a group of vectors in $\R^n$ with vector addition.
\end{definition}
\begin{definition}\label{EuclideanGroup}
    An euclidean group $E\left(n\right)$ is a semidirect product of orthogonal group $O\left(n\right)$ extended by translation group $T\left(n\right)$.
\end{definition}

\begin{definition}\label{KleinGeometry}
    A \textit{Klein geometry} is a pair $\left(G, H\right)$
    where $G$ is a Lie group
    and $H$ is a closed Lie subgroup of $G$
    such that the (left) coset space $$X\defeq G / H$$ is connected.
\end{definition}
\begin{example}\label{KleinGeometryExample}
    KleinGeometryExamples (I)
\end{example}

\subsection{Manifold}

\begin{definition}[Abstract differentiable manifold {\autocite[][Chapter~5A]{kuhnelwolfgang_2006}}]\label{Manifold}
    A \textit{$k$-dimensional differentiable manifold} (briefly: a $k$-manifold)
    is a set $M$ together with a family $\left(M_i\right)_{i\in I}$ of subsets such that
    \begin{APAenumerate}
        \item $M=\bigcup_{i\in I} M_i$ (union),
        \item for every $i\in I$ there is an injective map $\varphi_i:M_i\to\R^k$ so that $\phi_i\left(M_i\right)$ is open in $\R^k$, and
        \item for $M_i\cap M_j\ne\emptyset$, $\varphi_i\left(M_i\cap M_j\right)$ is open in $\R^k$.
    \end{APAenumerate}
\end{definition}

\begin{definition}[Structures on a manifold {\autocite[][Chapter~5A]{kuhnelwolfgang_2006}}]\label{Manifold:Extended}
    Given a $k$-dimensional differentiable manifold,
    one gets additional structure
    by replacing aditional requirements on the transformation functions $\varphi_j\circ\varphi_i^{-1}$,
    which belong to the atlas of the manifold;
    if all $\varphi_j\circ\varphi_i^{-1}$ are (left-hand side),
    then one speaks of (right-hand side) as follows:
    \begin{center}
        \begin{tabular}{ r c l }
            continuous                & $\leftrightarrow$ & topological manifold                                 \\
            differentiable            & $\leftrightarrow$ & differentiable manifold                              \\
            $C^1$-differentiable      & $\leftrightarrow$ & $C^1$-manifold                                       \\
            $C^r$-differentiable      & $\leftrightarrow$ & $C^r$-manifold                                       \\
            $C^\infty$-differentiable & $\leftrightarrow$ & $C^\infty$-manifold                                  \\
            real analytic             & $\leftrightarrow$ & real analytic manifold                               \\
            complex analytic          & $\leftrightarrow$ & complex analytic manifold of dimension $\frac{k}{2}$ \\
            affine                    & $\leftrightarrow$ & affine manifold                                      \\
            projective                & $\leftrightarrow$ & projective manifold                                  \\
            conformal                 & $\leftrightarrow$ & manifold with a conformal structure                  \\
            orienatation-preserving   & $\leftrightarrow$ & orientable manifold                                  \\
        \end{tabular}
    \end{center}
\end{definition}

\begin{definition}[Tangent vector {\autocite[][Chapter~5B]{kuhnelwolfgang_2006}}]\label{Manifold:TangentVector}
    A \textit{tangent vector} $X$ at $p$
    is a derivation (derivative operator) defined on the set of \textit{germs of functions}
    $$\mathcal{F}_p\left(M\right)\defeq\set{f:M\to\R|f\text{ differentiable}}/\sim\text{,}$$
    where the equivalence relation $\sim$ is defined by
    declaring $f\sim f^\ast$ if and only if
    $f$ and $f^\ast$ coincide in a neighborhood of $p$.
    The value $X\left(f\right)$ is also referred to as the \textit{directional derivative} of $f$ in the direction $X$.

    This definition means more precisely the following.
    $X$ is a map $X:\mathcal{F}_p\left(M\right)\to\R$
    with the two following properties:
    \begin{APAenumerate}
        \item $X\left(\alpha f+\beta g\right)=\alpha X\left(f\right)+\beta\left(g\right)$, $f,g\in\mathcal{F}_p\left(M\right)$ (\textit{$\R$-linearity});
        \item $X\left(f\cdot g\right)=X\left(f\right)\cdot g\left(p\right)+f\left(p\right)\cdot X\left(g\right)$ for $f,g\in\mathcal{F}_p\left(M\right)$ (\textit{product rule}).
    \end{APAenumerate}
    (For this to make sense, both $f$ and $g$ have to be defined in a neighborhood of $p$.)

    Briefly: \textit{tangent vectors are derivations acting on scalar functions.}
\end{definition}

\begin{corollary}[Tangent space {\autocite[][Chapter~5B]{kuhnelwolfgang_2006}}]\label{Manifold:TangentSpace}
    The \textit{tangent space} $T_pM$ of $M$ at $p$
    is defined in all cases as
    the set of all tangent vectors at the point $p$.
    By definition $T_pM$ and $T_qM$ are disjoint if $p\ne q$.
\end{corollary}

\begin{definition}[Derivative {\autocite[][Chapter~5B]{kuhnelwolfgang_2006}}]\label{Manifold:Derivative}
    Let $F:M\to N$ be a differentiable map,
    and let $p$, $q$ be two fixed point with $F\left(p\right)=q$.
    Then the \textit{derivative} or the $\textit{differential}$ of $F$ at $p$ is defined as the map
    $$\left.DF\right|_p:T_pM\to T_qN$$
    whose value at $X\in T_pM$ is given by
    $\left(\left.DF\right|_p\left(X\right)\right)\left(f\right)=X\left(f\circ F\right)$
    for every $f\in \mathcal{F}_q\left(N\right)$
    (which automatically implies the relation $f\circ F\in\mathcal{F}_p\left(M\right)$).
\end{definition}

\begin{lemma}[Chain rule {\autocite[][Chapter~5B]{kuhnelwolfgang_2006}}]\label{Manifold:ChainRule}
    For the derivative as defined in this manner, one has the \textit{chain rule} in the form
    $$\left.D\left(G\circ F\right)\right|_p=\left.DG\right|_{F\left(p\right)}\circ\left.DF\right|_p$$
    for every composition $M\xrightarrow{F}N\xrightarrow{G}Q$ of maps, or, more briefly, $D\left(G\circ F\right)=DG\circ DF$.
\end{lemma}

\begin{definition}[Riemannian metric {\autocite[][Chapter~5C]{kuhnelwolfgang_2006}}]\label{Manifold:RiemannianMetric}
    A \textit{Riemannian metric} $g$ on $M$
    is an association $p\mapsto g_p\in L^2\left(T_pM;\R\right)$
    such that the following conditins are satisfied:
    \begin{APAenumerate}
        \item $g_p\left(X,Y\right)=g_p\left(Y,X\right)$ for all $X$, $Y$, \hfill (\textit{symmetry})
        \item $g_p\left(X,X\right)>0$ for all $X\ne0$, \hfill (\textit{positive definiteness})
        \item The coefficient $\tensor{g}{_i_j}$ in every local representation (i.e., in every chart) $$g_p=\sum_{i,j}\tensor{g}{_i_j}\left(p\right)\cdot \left.d\tensor{x}{^i}\right|_p \otimes\left.d\tensor{x}{^j}\right|_p$$ are differentiable functions. \hfill (\textit{differentiability})
    \end{APAenumerate}
\end{definition}

\begin{remark}\label{Manifold:Riemannian}
    The pair $\left(M,g\right)$ is then called \textit{Riemannian manifold}.
    One also refers to the Riemannian metric as the \textit{metric tensor}.
    In local coordinates the metric tensor is given by the matrix ($\tensor{g}{_i_j}$) of functions.
    In Ricci calculus this is simply written as $\tensor{g}{_i_j}$.
        {\autocite[][Chapter~5C]{kuhnelwolfgang_2006}}
\end{remark}

\begin{remark}\label{Manifold:InnerProduct}
    A Riemannian metric $g$ defines at every point $p$
    an \textit{inner product} $g_p$ on the tangent space $T_pM$,
    and therefore the notation $\inner{X}{Y}$ instead of $g_p\left(X,Y\right)$ is also used.
    The notions of angles and lengths are determined by this inner product,
    just as these notions are determined by the first fundamental form on surface elements.
    The length or norm of vector $X$ is given by $\norm{X}\defeq\sqrt{g\left(X,X\right)}$,
    and the angle $\beta$ between two tangent vectors $X$ and $Y$
    can be defined by the validity of the equation $\cos\beta\cdot\norm{X}\cdot\norm{Y}=g\left(X,Y\right)$.
        {\autocite[][Chapter~5C]{kuhnelwolfgang_2006}}
\end{remark}

\subsection{Curvature}
\begin{definition}[Christoffel symbols {\autocite[][Chapter~4A]{kuhnelwolfgang_2006}}]\label{Manifold:ChristoffelSymbol}

    \begin{APAenumerate}
        \item The quantities $\partial_k\tensor{\Gamma}{_i_j}$ defined by the expressions
        \begin{equation*}
            \partial_k\tensor{\Gamma}{_i_j}
            \defeq
            \inner{\nabla_{\frac{\partial f}{\partial\tensor{u}{^i}}}\frac{\partial f}{\partial\tensor{u}{^j}}}{\frac{\partial f}{\partial\tensor{u}{^k}}}
        \end{equation*}
        are called the \textit{Christoffel symbols of the first kind}.
        \item The quantities $\tensor{\Gamma}{^k_i_j}$ defined by the expressions
        \begin{equation*}
            \nabla_{\frac{\partial f}{\partial\tensor{u}{^i}}}\frac{\partial f}{\partial\tensor{u}{^j}}
            =
            \sum_k{\tensor{\Gamma}{^k_i_j}\frac{\partial f}{\partial\tensor{u}{^k}}}
        \end{equation*}
        are called the \textit{Christoffel symbols of the second kind}.
        \item By definition one has $\partial_k\tensor{\Gamma}{_i_j}=\partial_k\tensor{\Gamma}{_j_i}$, $\tensor{\Gamma}{^k_i_j}=\tensor{\Gamma}{^k_j_i}$
        as well as $\partial_k\tensor{\Gamma}{_i_j}=\sum_{m}\tensor{\Gamma}{^m_i_j}\tensor{g}{_m_k}$.
    \end{APAenumerate}
\end{definition}
\begin{definition}[Gauss map {\autocite[][Chapter~3B]{kuhnelwolfgang_2006}}]\label{Curvature:Gauss}
    For a surface element $f:U\to\R^{3}$, the \textit{Gauss map}
    \begin{equation*}
        \nu:U\to S^2
    \end{equation*}
    is defined by the formula
    \begin{equation*}
        \nu(u_1, u_2)
        \defeq
        \frac
        {\frac{\partial f}{\partial u_1}\times\frac{\partial f}{\partial u_2}}
        {\norm{\frac{\partial f}{\partial u_1}\times\frac{\partial f}{\partial u_2}}}
        \text{.}
    \end{equation*}
\end{definition}
\begin{definition}[Weingarten map, shape operator {\autocite[][Chapter~3B]{kuhnelwolfgang_2006}}]\label{Curvature:ShapeOperator}
    Let $f:U\to\R^{3}$ be a surface element with Gauss map $\nu:U\to S^2\subset\R^3$.
    \begin{APAenumerate}
        \item For every $u\in U$ the image plane of the bilinear map
        \begin{equation*}
            \left.D\nu\right|_u:T_uU\to T_{\nu(u)}\R^3
        \end{equation*}
        is parallel to the tangent plane $T_uf$.
        By canonically identifying $T_{\nu(u)}\R^3\cong\R^3\cong T_{f(u)}\R^3$
        we may therefore view $D\nu$ at every point as the map
        \begin{equation*}
            \left.D\nu\right|_u:T_uU\to T_uf \text{.}
        \end{equation*}
        Moreover, by restricting to the image,
        we may view the map $\left.Df\right|)u$ as a linear isomorphism
        \begin{equation*}
            \left.Df\right|_u:T_uU\to T_uf \text{.}
        \end{equation*}
        In this sense
        the inverse mapping $\left(\left.Df\right|_u\right)^{-1}$ is well-defined
        and is also an isomorphism
        \item The map $L\defeq -D\nu\circ\left(Df\right)^{-1}$ defined pointwise by
        \begin{equation*}
            L_u
            \defeq -\left(\left.D\nu\right|_u\right)\circ\left(\left.Df\right|_u\right)^{-1}
            : T_uf\to T_uf
        \end{equation*}
        is called the \textit{Weingarten map} or the \textit{shape operator} of $f$.
        Obviously, for every parameter $u$ this is a linear endomorphism of the tangent plane at the corresponding point $f(u)$.
        \item $L$ is independent of the parameterization $f$ (up to the choice of the sign of the unit normal vector $\nu$),
        and it is self-adjoint with respect to the first fundamental form $I$.
    \end{APAenumerate}
\end{definition}
\begin{definition}[Hypersurface element {\autocite[][Chapter~3F]{kuhnelwolfgang_2006}}]\label{Manifold:Hypersurface}
    $f:U\to\R^{n+1}$ is called a \textit{regular hypersurface element},
    if $U\subset\R^n$ is open and $f$ is a ($C^2-$) immersion.
    The parameter $u=\left(u_1,\dots,u_n\right)$ is associated with the point $f(u)$
    with $n+1$ coordinates $f(u)=\left(f_1(u),\dots,f_{n+1}(u)\right)$.
    The \textit{tangent hyperplane} $T_uf$ is the is defined to be the image of $T_uU$ under the map $\left.Df\right|_u$.
    Similarly, one defines
    \begin{APAitemize}
        \item the \textit{Gauss map} $\nu:U\to S^n$ by the unit normal vector $\nu(u)$,
        which is perpendicular to $T_uf$
        (
        but note: in $\R^{n+1}$ for $n\ge3$ there is no bilinear vector product of tangent vectors;
        still one can formally define $\nu$ as an $n$-linear vector product
        ),
        \item the \textit{Weingarten map} $L=-D\nu\circ\left(Df\right)^{-1}$,
        \item the \text{first, second, and third fundamental forms}
        \begin{align*}
            I   & = \left(\tensor{g}{_i_j}\right)_{i,j=1,\dots,n} & = \left(\inner{\frac{\partial f}{\partial u_i}}{\frac{\partial f}{\partial u_j}}\right)\text{,}     \\
            II  & = \left(\tensor{h}{_i_j}\right)_{i,j=1,\dots,n} & = \left(\inner{\frac{\partial^2 f}{\partial u_i\partial u_j}}{\nu}\right)\text{,}                   \\
            III & = \left(\tensor{e}{_i_j}\right)_{i,j=1,\dots,n} & = \left(\inner{\frac{\partial \nu}{\partial u_i}}{\frac{\partial \nu}{\partial u_j}}\right)\text{.} \\
        \end{align*}
    \end{APAitemize}
\end{definition}
\begin{definition}[Curvature tensor {\autocite[][Chapter~4C]{kuhnelwolfgang_2006}}]\label{Manifold:CurvatureTensor}
    \begin{equation*}
        R\left(X,Y\right)Z
        \defeq
        \nabla_X\nabla_YZ - \nabla_Y\nabla_XZ - \nabla_{\lie{X}{Y}}Z
    \end{equation*}
    is a tensor field, which is called the \textit{curvature tensor} of the surface.
\end{definition}
\begin{remark}[Curvature tensor {\autocite[][Chapter~4C]{kuhnelwolfgang_2006}}]\label{Manifold:CurvatureTensor:ChristoffelSymbol}
    The left-hand side of the Gauss equation is called the \textit{curvature tensor}
    and is in general expressed in the form
    \begin{equation*}
        \tensor{R}{^s_i_k_j}
        \defeq
        \frac{\partial}{\partial\tensor{u}{^k}}\tensor{\Gamma}{^s_i_j}
        - \frac{\partial}{\partial\tensor{u}{^j}}\tensor{\Gamma}{^s_i_k}
        + \sum_{r}\left(\tensor{\Gamma}{^r_i_j}\tensor{\Gamma}{^s_r_k}-\tensor{\Gamma}{^r_i_k}\tensor{\Gamma}{^s_r_j}\right) \text{.}
    \end{equation*}
\end{remark}

\begin{example}\label{ShapeOperator}
    $$S\left(v\right)=\pm\nabla_v n$$
\end{example}
\begin{example}\label{PrincipalCurvature}
    Eigenvalue of second fundamental form (or shape operator)
\end{example}
\end{document}