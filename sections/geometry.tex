% !TeX root = ../main.tex

\documentclass[../main.tex]{subfiles}

\begin{document}
\section{Geometric properties}
\subsection{Embedding}
\begin{definition}\label{M:Embedding}
    Let $N=\left(Q,g\right)$ be a $n$-dimension Riemannian manifold
    on position parameter space with such inner product $g$ that
    the map $\cdot\mapsto\frac{1}{k}\cdot\tensor{e}{^1}\in P\to P^\ast$ is an isometric embedding to $\left(n+1\right)$-Euclidean manifold.
\end{definition}
\begin{remark}
    It have been proved that there is no isometric embedding from $H^n$ to $E^{n+1}$.
    Hence, this section is based on a faulty assumption and would not provide a useful information.
\end{remark}
\begin{lemma}\label{M:Tangent:Basis}
    Position parameter $\tensor{\theta}{^i}$ is associated to the following vector in position vector space.
    \begin{equation*}
        \frac{\partial}{\partial\tensor{\theta}{^i}} =
        \begin{pmatrix}
            -k\tan_k^\ast\tensor{\theta}{^i}\tensor{p}{^j}         \\
            k\frac{1}{\tan_k\tensor{\theta}{^i}}\tensor{p}{^{i+1}} \\
            0
        \end{pmatrix} \frac{\partial}{\partial\tensor{p}{^j}}
    \end{equation*}
\end{lemma}
\begin{proof}[\proofof{M:Tangent:Basis}]
    \begin{align*}
        \frac{\partial}{\partial\tensor{\theta}{^i}}
                                                                                                        & = \frac{\partial\tensor{p}{^j}}{\partial\tensor{\theta}{^i}}\frac{\partial}{\partial\tensor{p}{^j}} \\
                                                                                                        & = \frac{\partial}{\partial\tensor{\theta}{^i}}
        \left[
            \frac{1}{k}
            \begin{pmatrix}
                \prod_{l\in\Set{1..n}}{\cos_k{\tensor{\theta}{^l}}}                                \\
                \sin_k{\tensor{\theta}{^{j-1}}}\prod_{l\in\Set{j..n}}{\cos_k{\tensor{\theta}{^l}}} \\
                \sin_k{{\tensor{\theta}{^n}}}                                                      \\
            \end{pmatrix}
        \right]\frac{\partial}{\partial\tensor{p}{^j}}                                                  & \text{\cref{M:Vector:Value}}                                                                        \\
                                                                                                        & =
        \frac{1}{k}
        \frac{\partial}{\partial\tensor{\theta}{^i}}
        \left[
            \begin{pmatrix}
                \prod_{l\in\Set{1..n}}{\cos_k{\tensor{\theta}{^l}}}                                \\
                \sin_k{\tensor{\theta}{^{j-1}}}\prod_{l\in\Set{j..n}}{\cos_k{\tensor{\theta}{^l}}} \\
                \sin_k{{\tensor{\theta}{^n}}}                                                      \\
            \end{pmatrix}
        \right]\frac{\partial}{\partial\tensor{p}{^j}}                                                                                                                                                        \\
                                                                                                        & =
        \frac{1}{k}
        \left[
            \begin{pmatrix}
                -k\sin_k^\ast{\tensor{\theta}{^i}}\prod_{l\in\Set{1..n}/\Set{i}}{\cos_k{\tensor{\theta}{^l}}}                                \\
                -k\sin_k^\ast{\tensor{\theta}{^i}}\sin_k{\tensor{\theta}{^{j-1}}}\prod_{l\in\Set{j..n}/\Set{i}}{\cos_k{\tensor{\theta}{^l}}} \\
                k\cos_k{\tensor{\theta}{^{i}}}\prod_{l\in\Set{i+1..n}}{\cos_k{\tensor{\theta}{^l}}}                                          \\
                0                                                                                                                            \\
            \end{pmatrix}
        \right]\frac{\partial}{\partial\tensor{p}{^j}}                                                  & \text{\cref{M:Trigonometry:Derivative}}                                                             \\
                                                                                                        & =
        \begin{pmatrix}
            -\tan_k^\ast{\tensor{\theta}{^i}}\prod_{l\in\Set{1..n}}{\cos_k{\tensor{\theta}{^l}}}                                \\
            -\tan_k^\ast{\tensor{\theta}{^i}}\sin_k{\tensor{\theta}{^{j-1}}}\prod_{l\in\Set{j..n}}{\cos_k{\tensor{\theta}{^l}}} \\
            \frac{1}{\tan_k{\tensor{\theta}{^{i}}}}\prod_{l\in\Set{i+1..n}}{\cos_k{\tensor{\theta}{^l}}}                        \\
            0                                                                                                                   \\
        \end{pmatrix}
        \frac{\partial}{\partial\tensor{p}{^j}}                                                                                                                                                               \\
                                                                                                        & =
        \begin{pmatrix}
            -k\tan_k^\ast\tensor{\theta}{^i}\tensor{p}{^j}         \\
            k\frac{1}{\tan_k\tensor{\theta}{^i}}\tensor{p}{^{i+1}} \\
            0
        \end{pmatrix} \frac{\partial}{\partial\tensor{p}{^j}} & \text{\cref{M:Vector:Value}}                                                                                                          \\
    \end{align*}
\end{proof}
\begin{lemma}\label{M:MetricTensor}
    The metric tensor of $N$ is
    \begin{align*}
        \tensor{g}{_i_j} & =
        \begin{cases}
            \left(1-\sign{k}\right)
            \tan_k^2\left(\tensor{\theta}{^a}\right)
            \prod_{1\le j}{\cos_k^2\tensor{\theta}{^j}}
            +
            \prod_{a<j}{\cos_k^2\tensor{\theta}{^j}} & \text{if $i=j$,}  \\
            0                                        & \text{otherwise.} \\
        \end{cases}
    \end{align*}
\end{lemma}
\begin{proof}[\proofof{M:MetricTensor}]
    \begin{align*}
        \tensor{g}{_a_b}\left[\frac{\partial}{\partial\tensor{\theta}{^i}}\right]
         & = \sum_{l,m=1}^{n+1}{
        \frac{\partial{\tensor{p}{^l}}}{\partial{\tensor{\theta}{^a}}}
        \tensor{g}{_l_m}\left[\frac{\partial}{\partial\tensor{p}{^i}}\right]
        \frac{\partial{\tensor{p}{^m}}}{\partial{\tensor{\theta}{^b}}}
        }                        \\
         & = \sum_{l=1}^{n+1}{
        \frac{\partial{\tensor{p}{^l}}}{\partial{\tensor{\theta}{^a}}}
        \frac{\partial{\tensor{p}{^l}}}{\partial{\tensor{\theta}{^b}}}
        }
    \end{align*}

    If $a<b$,
    \begin{align*}
        \tensor{g}{_a_b}
                         & =\begin{bmatrix}
                                -k\tan_k^\ast\left(\tensor{\theta}{^a}\right)\tensor{p}{^j}         \\
                                k\frac{1}{\tan_k\left(\tensor{\theta}{^a}\right)}\tensor{p}{^{a+1}} \\
                                0                                                                   \\
                            \end{bmatrix}\cdot\begin{bmatrix}
                                                  -k\tan_k^\ast\left(\tensor{\theta}{^b}\right)\tensor{p}{^j}         \\
                                                  k\frac{1}{\tan_k\left(\tensor{\theta}{^b}\right)}\tensor{p}{^{b+1}} \\
                                                  0                                                                   \\
                                              \end{bmatrix} \\
                         & =
        \sum{
            k^2\tan_k^\ast\left(\tensor{\theta}{^a}\right)\tan_k^\ast\left(\tensor{\theta}{^b}\right){\tensor{p}{^j}}^2
        }
        -k^2\frac{\tan_k^\ast\left(\tensor{\theta}{^b}\right)}{\tan_k\left(\tensor{\theta}{^a}\right)}{\tensor{p}{^{a+1}}}^2               \\
                         & =
        \tan_k\left(\tensor{\theta}{^b}\right)
        \tan_k\left(\tensor{\theta}{^a}\right)
        \prod_{a\le j\le n+1}{\cos_k^2\tensor{\theta}{^j}}
        \left(
        \prod_{1\le j<a}{\cos_k^2\tensor{\theta}{^j}}
        + \sum{
            \sin_k^2{\tensor{\theta}{^{i}}}\prod_{i<j<a}{\cos_k^2\tensor{\theta}{^j}}
        }
        -\sign{k}
        \right)                                                                                                                            \\
                         & =
        \tan_k\left(\tensor{\theta}{^b}\right)
        \tan_k\left(\tensor{\theta}{^a}\right)
        \prod_{a\le j\le n+1}{\cos_k^2\tensor{\theta}{^j}}
        \left(
        \prod_{1\le j<a}{\cos_k^2\tensor{\theta}{^j}}
        + \sign{k}\left(
        1-\prod_{1\le j<a}{\cos_k^2\tensor{\theta}{^j}}
        \right)
        - \sign{k}
        \right)                                                                                                                            \\
        \tensor{g}{_a_b} & = 0 \text{.}
    \end{align*}

    If $a>b$, $\tensor{g}{_a_b}=\tensor{g}{_b_a}=0$.

    If $a=b$,
    \begin{align*}
        \tensor{g}{_a_a}
         & =\begin{pmatrix}
                -k\tan_k^\ast\tensor{\theta}{^a}\tensor{p}{^j}         \\
                k\frac{1}{\tan_k\tensor{\theta}{^a}}\tensor{p}{^{a+1}} \\
                0
            \end{pmatrix}\cdot\begin{pmatrix}
                                  -k\tan_k^\ast\tensor{\theta}{^a}\tensor{p}{^j}         \\
                                  k\frac{1}{\tan_k\tensor{\theta}{^a}}\tensor{p}{^{a+1}} \\
                                  0
                              \end{pmatrix} \\
         & =
        \sum_{j=1}^{a}{\left(k\tensor{p}{^j}\tan_k^\ast\tensor{\theta}{^a}\right)^2}
        +\left(k\tensor{p}{^{a+1}}\cot_k\tensor{\theta}{^a}\right)^2                                                       \\
         & =
        \sum_{j=1}^{a}{\left(k\tensor{p}{^j}\tan_k\tensor{\theta}{^a}\right)^2}
        +\left(k\tensor{p}{^{a+1}}\cot_k\tensor{\theta}{^a}\right)^2                                                       \\
         & =
        \sin_k^2\left(\tensor{\theta}{^a}\right)
        \prod_{a<j}{\cos_k^2\tensor{\theta}{^j}}
        \left(
        \prod_{1\le j<a}{\cos_k^2\tensor{\theta}{^j}}
        + \sum_{1\le i<a}{
            \sin_k^2{\tensor{\theta}{^{i}}}\prod_{i<j<a}{\cos_k^2\tensor{\theta}{^j}}
        }
        +\cot_k^2\left(\tensor{\theta}{^a}\right)
        \right)                                                                                                            \\
         & =
        \sin_k^2\left(\tensor{\theta}{^a}\right)
        \prod_{a<j}{\cos_k^2\tensor{\theta}{^j}}
        \left(
        \prod_{1\le j<a}{\cos_k^2\tensor{\theta}{^j}}
        + \sum_{1\le i<a}{
            \sin_k^2{\tensor{\theta}{^{i}}}\prod_{i<j<a}{\cos_k^2\tensor{\theta}{^j}}
        }
        +\cot_k^2\left(\tensor{\theta}{^a}\right)
        \right)                                                                                                            \\
         & =
        \sin_k^2\left(\tensor{\theta}{^a}\right)
        \prod_{a<j}{\cos_k^2\tensor{\theta}{^j}}
        \left(
        \prod_{1\le j<a}{\cos_k^2\tensor{\theta}{^j}}
        -\sign{k}\prod_{1\le j<a}{\cos_k^2\tensor{\theta}{^j}}
        +\sign{k}\prod_{1\le j<a}{\cos_k^2\tensor{\theta}{^j}}
        + \sum_{1\le i<a}{
            \sin_k^2{\tensor{\theta}{^{i}}}\prod_{i<j<a}{\cos_k^2\tensor{\theta}{^j}}
        }
        +\cot_k^2\left(\tensor{\theta}{^a}\right)
        \right)                                                                                                            \\
         & =
        \sin_k^2\left(\tensor{\theta}{^a}\right)
        \prod_{a<j}{\cos_k^2\tensor{\theta}{^j}}
        \left(
        \prod_{1\le j<a}{\cos_k^2\tensor{\theta}{^j}}
        -\sign{k}\prod_{1\le j<a}{\cos_k^2\tensor{\theta}{^j}}
        +\sign{k}
        +\cot_k^2\left(\tensor{\theta}{^a}
        \right)
        \right)                                                                                                            \\
         & =
        \sin_k^2\left(\tensor{\theta}{^a}\right)
        \prod_{a<j}{\cos_k^2\tensor{\theta}{^j}}
        \left(
        \prod_{1\le j<a}{\cos_k^2\tensor{\theta}{^j}}
        -\sign{k}\prod_{1\le j<a}{\cos_k^2\tensor{\theta}{^j}}
        +\sign{k}
        +\csc_k^2\left(\tensor{\theta}{^a}
        -\sign{k}
        \right)
        \right)                                                                                                            \\
         & =
        \sin_k^2\left(\tensor{\theta}{^a}\right)
        \prod_{a<j}{\cos_k^2\tensor{\theta}{^j}}
        \left(
        \left(1-\sign{k}\right)\prod_{1\le j<a}{\cos_k^2\tensor{\theta}{^j}}
        +\csc_k^2\left(\tensor{\theta}{^a}
        \right)
        \right)                                                                                                            \\
         & =
        \left(1-\sign{k}\right)
        \tan_k^2\left(\tensor{\theta}{^a}\right)
        \prod_{j}{\cos_k^2\tensor{\theta}{^j}}
        +
        \prod_{a<j}{\cos_k^2\tensor{\theta}{^j}}
    \end{align*}
\end{proof}
\subsection{Curvature}
\subsubsection{Curvature (Method I)}
This method may be easier to generalize to
Model II where each direction can have partially independent curvature
(or even Model III where extrinsic curvature become a thing).
But it may be challenging to define Gauss map properly.
\begin{lemma}\label{M:Normal}
    Given a position parameter $\tensor{\theta}{}$,
    the tangent vector in position vector space can be calculated as follows
    \begin{equation*}
        \nu(p)
        =
        \begin{cases}
            \begin{bmatrix}
                k\tensor{p}{^1}  \\
                +k\tensor{p}{^i} \\
            \end{bmatrix} & k>0     \\
            \begin{bmatrix}
                1 \\
                0 \\
            \end{bmatrix}     & k=0 \\
            \frac{1}{\sqrt{-1+2\left(k\tensor{p}{^1}\right)^2}}
            \begin{bmatrix}
                k\tensor{p}{^1}  \\
                -k\tensor{p}{^i} \\
            \end{bmatrix} & k<0
        \end{cases}
        \text{.}
    \end{equation*}
\end{lemma}
\begin{proof}[\proofof{M:Normal}]
    From \cref{SphericalLocus,EuclideanLocus,HyperbolicLocus},
    \begin{equation*}
        p \in P \iff
        \begin{cases}
           \sum_{i}{{\tensor{p}{^i}}^2}=k^{-2} & \text{$k>0$,} \\
           \tensor{p}{^1} = k^{-1} \iffalse{\wedge \tensor{p}{^i} = \tensor{\theta}{^{i-1}}}\fi & \text{$k\to 0$,} \\
             {\tensor{p}{^1}}^2 - \sum_{1<i}{{\tensor{p}{^i}}^2}=k^{-2} \wedge \tensor{p}{^1} > 0 & \text{$k<0$.} \\
        \end{cases}
    \end{equation*}
    Let
    \begin{equation*}
        F(p) = \begin{cases}
           \sum_{i}{{\tensor{p}{^i}}^2}-k^{-2} & \text{$k>0$,} \\
           \tensor{p}{^1} - k^{-1} & \text{$k\to 0$,} \\
             {\tensor{p}{^1}}^2 - \sum_{1<i}{{\tensor{p}{^i}}^2}-k^{-2} & \text{$k<0$.} \\
        \end{cases}
    \end{equation*}
    \begin{align*}
        n
        &= l \nabla F(p) \\
        &= l \begin{cases}
            \nabla{\sum_{i}{{\tensor{p}{^i}}^2}-k^{-2}} & \text{$k>0$,} \\
            \nabla{\tensor{p}{^1} - k^{-1}} & \text{$k\to 0$,} \\
            \nabla{{\tensor{p}{^1}}^2 - \sum_{1<i}{{\tensor{p}{^i}}^2}-k^{-2}} & \text{$k<0$.} \\
         \end{cases} \\
         &= l \begin{cases}
            \begin{pmatrix}
                2\tensor{p}{^i} \\
            \end{pmatrix} & \text{$k>0$,} \\
            \begin{pmatrix}
                1 \\
                0 \\
            \end{pmatrix} & \text{$k\to 0$,} \\
            \begin{pmatrix}
                -2\tensor{p}{^1} \\
                2\tensor{p}{^i} \\
            \end{pmatrix} & \text{$k<0$.} \\
          \end{cases} \\
          \hat{n}
          &= \pm
            \begin{cases}
            \frac{1}{\norm{p}}
             \begin{pmatrix}
                 \tensor{p}{^i} \\
             \end{pmatrix} & \text{$k>0$,} \\
             \begin{pmatrix}
                 1 \\
                 0 \\
             \end{pmatrix} & \text{$k\to 0$,} \\
             \frac{1}{\norm{p}}
             \begin{pmatrix}
                 -\tensor{p}{^1} \\
                 \tensor{p}{^i} \\
             \end{pmatrix} & \text{$k<0$.} \\
           \end{cases} \\
           &= \pm
             \begin{cases}
             \frac{1}{k^{-1}}
              \begin{pmatrix}
                  \tensor{p}{^i} \\
              \end{pmatrix} & \text{$k>0$,} \\
              \begin{pmatrix}
                  1 \\
                  0 \\
              \end{pmatrix} & \text{$k\to 0$,} \\
              \frac{1}{\sqrt{-1+2\left(k\tensor{p}{^1}\right)^2}}
              \begin{pmatrix}
                  -\tensor{p}{^1} \\
                  \tensor{p}{^i} \\
              \end{pmatrix} & \text{$k<0$.} \\
            \end{cases} \\
            \nu=\hat{n}
            &\defeq
              \begin{cases}
               \begin{pmatrix}
                   k\tensor{p}{^i} \\
               \end{pmatrix} & \text{$k>0$,} \\
               \begin{pmatrix}
                   1 \\
                   0 \\
               \end{pmatrix} & \text{$k\to 0$,} \\
               \frac{1}{\sqrt{-1+2\left(k\tensor{p}{^1}\right)^2}}
               \begin{pmatrix}
                   \tensor{p}{^1} \\
                   -\tensor{p}{^i} \\
               \end{pmatrix} & \text{$k<0$.} \\
             \end{cases} \qedhere
    \end{align*}
\end{proof}
\begin{lemma}\label{M:ShapeOperator}
    \begin{equation*}
        L(p) = \begin{cases}
            -k I & \text{$k>0$,} \\
            0 & \text{$k\to0$,} \\
            \frac{1}{\norm{p}} \left(\frac{1}{\norm{p}^2} p p^T \diag{\left(-1,1\dots,1\right)} - I\right) & \text{$k<0$.} \\
        \end{cases}
    \end{equation*}
\end{lemma}
\begin{proof}[\proofof{M:ShapeOperator}]
    \begin{align*}
        \frac{\partial}{\partial\tensor{p}{^j}}{\frac{1}{\norm{p}}}
        &= \frac{\partial}{\partial\norm{p}}{\frac{1}{\norm{p}}}\frac{\partial}{\partial\tensor{p}{^j}}{\norm{p}} \\
        &= {-\frac{1}{\norm{p}^2}}\frac{\partial}{\partial\tensor{p}{^j}}{\sqrt{\sum_i{{\tensor{p}{^i}}^2}}} \\
        &= {-\frac{1}{\norm{p}^2}}\frac{\partial}{\partial \sum_i{{\tensor{p}{^i}}^2}}{\sqrt{\sum_i{{\tensor{p}{^i}}^2}}}\frac{\partial}{\partial\tensor{p}{^j}}{\sum_i{{\tensor{p}{^i}}^2}} \\
        &= {-\frac{1}{\norm{p}^2}}\frac{1}{2\sqrt{\sum_i{{\tensor{p}{^i}}^2}}}{2{{\tensor{p}{^j}}}} \\
        &= {-\frac{1}{\norm{p}^2}}\frac{1}{2\norm{p}}{2{{\tensor{p}{^j}}}} \\
        &= -\frac{\tensor{p}{^j}}{\norm{p}^3}
    \end{align*}
    From \cref{Manifold:Hypersurface},
    \begin{align*}
        L(p)
        &= - (D\nu \circ \left(Df\right)^{-1})(p) \\
        &= - (D\nu)(p) \\
        &= - \left(\left.\frac{\partial}{\partial\tensor{p}{^j}}\tensor{\nu}{^i}\right|_p\right)_{i,j} \\
        \left(L(p)\right)_{i,j}
        &= \begin{cases}
            - \frac{\partial}{\partial\tensor{p}{^j}} k\tensor{p}{^i} & \text{$k>0$,} \\
            - \frac{\partial}{\partial\tensor{p}{^j}} 1 & \text{$k\to 0$ and $i=1$,} \\
            - \frac{\partial}{\partial\tensor{p}{^j}} 0 & \text{$k\to 0$ and $i\ne1$,} \\
            - \frac{\partial}{\partial\tensor{p}{^j}} \left({\frac{1}{\norm{p}}}\tensor{p}{^1}\right) & \text{$k<0$ and $i=1$,} \\
            - \frac{\partial}{\partial\tensor{p}{^j}} \left(-{\frac{1}{\norm{p}}}\tensor{p}{^i}\right) & \text{$k<0$ and $i\ne 1$,} \\
          \end{cases} \\
          &= \begin{cases}
              - k & \text{$k>0$ and $i=j$,} \\
              0 & \text{$k>0$ and $i\ne j$,} \\
              0 & \text{$k\to 0$,} \\
              - \left({\frac{1}{\norm{p}}}\frac{\partial}{\partial\tensor{p}{^j}}\tensor{p}{^1} + \tensor{p}{^1}\frac{\partial}{\partial\tensor{p}{^j}}{\frac{1}{\norm{p}}}\right) & \text{$k<0$ and $i=1$,} \\
              \left({\frac{1}{\norm{p}}}\frac{\partial}{\partial\tensor{p}{^j}}\tensor{p}{^i} + \tensor{p}{^i}\frac{\partial}{\partial\tensor{p}{^j}}{\frac{1}{\norm{p}}}\right) & \text{$k<0$ and $i\ne 1$,} \\
            \end{cases} \\
            &= \begin{cases}
                - k & \text{$k>0$ and $i=j$,} \\
                0 & \text{$k>0$ and $i\ne j$,} \\
                0 & \text{$k\to 0$,} \\
                -{\frac{1}{\norm{p}}} + \frac{{\tensor{p}{^1}}^2}{\norm{p}^3} & \text{$k<0$ and $i=j=1$,} \\
                \frac{\tensor{p}{^1}\tensor{p}{^j}}{\norm{p}^3} & \text{$k<0$ and $j\ne i=1$,} \\
                {\frac{1}{\norm{p}}} - \frac{{\tensor{p}{^i}}^2}{\norm{p}^3} & \text{$k<0$ and $i=j\ne1$,} \\
                -\frac{\tensor{p}{^i}\tensor{p}{^j}}{\norm{p}^3} & \text{$k<0$, $i\ne j$ and $i\ne1$,} \\
              \end{cases}
    \end{align*}

    For $k>0$, $L(p)=-k I$.

    For $k\to0$, $L(p)=0$.

    For $k<0$, $L(p)= \frac{1}{\norm{p}} \left(\frac{1}{\norm{p}^2} p p^T \diag{\left(-1,1\dots,1\right)} - I\right)$.
\end{proof}
\begin{lemma}\label{Model:PrincipalCurvature}

\end{lemma}
\begin{proof}[\proofof{Model:PrincipalCurvature}]
    \skipped

    For $k>0$,
    \begin{align*}
        0
        &= \det{L(p)-\lambda I} \\
        &= \det{-k I-\lambda I} \\
        &= \det{(-k-\lambda) I} \\
        &= (-k-\lambda)^{n+1}\det{I} \\
        &= (-k-\lambda)^{n+1} \\
        \lambda &\in \Set{-k,\dots,-k}
    \end{align*}

    For $k<0$,
    \begin{align*}
        0
        &= \det{L(p)-\lambda I} \\
        &= \det{\frac{1}{\norm{p}} \left(\frac{1}{\norm{p}^2} p p^T \diag{\left(-1,1\dots,1\right)} - I\right)-\lambda I} \\
        &= \det{\frac{1}{\norm{p}} \left(\frac{1}{\norm{p}^2} p p^T \diag{\left(-1,1\dots,1\right)} - I\right)-\left(\lambda^\prime-\frac{1}{\norm{p}}\right) I} & \text{$\lambda^\prime \defeq \lambda + \frac{1}{\norm{p}}$}\\
        &= \det{\frac{1}{\norm{p}^3} p p^T \diag{\left(-1,1\dots,1\right)} - \frac{1}{\norm{p}}I-\left(\lambda^\prime-\frac{1}{\norm{p}}\right) I}\\
        &= \det{\frac{1}{\norm{p}^3} p p^T \diag{\left(-1,1\dots,1\right)} - \lambda^\prime I}\\
        &= \det{\frac{1}{\norm{p}^3} p p^T \diag{\left(-1,1\dots,1\right)} - \frac{1}{\norm{p}^3}\lambda^{\prime\prime} I} & \text{$\lambda^{\prime\prime} \defeq \norm{p}^3\lambda^\prime$}\\
        &= \frac{1}{\norm{p}^{3(n+1)}}\det{p p^T \diag{\left(-1,1\dots,1\right)} - \lambda^{\prime\prime} I}\\
        &= \det{p p^T \diag{\left(-1,1\dots,1\right)} - \lambda^{\prime\prime} I}\\
        &= {\lambda^{\prime\prime}}^n\left(\lambda^{\prime\prime}+\tensor{p}{^1}^2-\sum_{i\ne 1}{\tensor{p}{^i}^2}\right) \\
        0 &= {\lambda^{\prime\prime}}^n\left(\lambda^{\prime\prime}+\frac{1}{k}\right) \\
        \lambda^{\prime\prime} &\in \Set{-\frac{1}{k},0,\dots,0} \\
        \lambda^\prime &\in \Set{-\frac{1}{k\norm{p}^3},0,\dots,0} \\
        \lambda &\in \Set{-\frac{1}{k\norm{p}^3} - \frac{1}{\norm{p}},-\frac{1}{\norm{p}},\dots,-\frac{1}{\norm{p}}} \\
        &= \Set{-\frac{1+\norm{p}^2}{k\norm{p}^3},-\frac{1}{\norm{p}},\dots,-\frac{1}{\norm{p}}} \\
        &= \Set{-\frac{2k^2{\tensor{p}{^1}}^2}{k\norm{p}^3},-\frac{1}{\norm{p}},\dots,-\frac{1}{\norm{p}}} \\
        &= \Set{-\frac{2k{\tensor{p}{^1}}^2}{\norm{p}^3},-\frac{1}{\norm{p}},\dots,-\frac{1}{\norm{p}}} \\
        &= \Set{k^2\norm{p}\frac{-2{\tensor{p}{^1}}^2}{k\norm{p}^4},-\frac{1}{\norm{p}},\dots,-\frac{1}{\norm{p}}} \\
        &= \Set{k^2\norm{p}\frac{-2{\tensor{p}{^1}}^2}{k\norm{p}^4},-\frac{1}{\norm{p}},\dots,-\frac{1}{\norm{p}}} \\
    \end{align*}
\end{proof}
\subsubsection{Curvature (Method II)}
This method may be easier to be done
(despite the fact that it never finished).
But it raises problems when trying to generalize e.g. dealing with extrinsic curvature
(which may be introduced in Model III
if not to mess with other basis geometries).
\begin{lemma}\label{Model:ChristoffelSymbol}

\end{lemma}
\begin{lemma}\label{Model:RiemannCurvatureTensor}

\end{lemma}
\subsubsection{Curvature (Conclusion)}
\begin{lemma}\label{Model:SectionalCurvature}

\end{lemma}
\paragraph{CurvatureParameter}
It can be seen that $\operatorname{sec}\left(p\right) = \kappa = \sign{\left(k\right)}k^2$.
Hence, when provided $\kappa$, $k$ can be determined and used to evaluate the model.

\end{document}