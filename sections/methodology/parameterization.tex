% !TeX root = ..\..\main.tex
\documentclass[../methodology.tex]{subfiles}
\begin{document}
\subsection{Model Parametrization}
\begin{definition}\label{M:Parameter}
    For any point matrix \(\tensor{X^\pm_{\lambda,n}\left(\theta,\phi_1,\phi_2,\dots,\phi_n\right)}{}\),
    \(n\)-dimensional vector \(\tensor{\theta}{}\)
    is defined as \textit{position parameter}.
\end{definition}
\begin{definition}\label{M:Vector}
    For point matrix \(\tensor{X}{}\),
    \(\left(n+1\right)\)-dimensional column vector \(\tensor{p}{}\coloneqq\frac{1}{\lambda}\tensor{X}{}\cdot\tensor{e}{^1}=\frac{1}{\lambda}\tensor{X}{_1}\)
    is defined as \textit{position vector}.
\end{definition}
\begin{lemma}\label{M:Vector:Position}
    For point matrix \(X=PO\)
    where \(P\) and \(O\) are position and orientation matrix respectively,
    \(\tensor{p}{}=\frac{1}{\lambda}\tensor{X}{_1}=\frac{1}{\lambda}\tensor{P}{_1}\).
\end{lemma}
\begin{proof}[Proof of \cref{M:Vector:Position}]
    \begin{align*}
        \tensor{p}{^i}
         & =\frac{1}{\lambda}\tensor{X}{^i_1}                                       \\
         & =\frac{1}{\lambda}\sum_j{\tensor{P}{^i_j}\tensor{O}{^j_1}}               \\
         & =\frac{1}{\lambda}\tensor{P}{^i_1}                                       \\
        \tensor{p}{}
         & =\frac{1}{\lambda}\tensor{P}{_1}                           &  & \qedhere
    \end{align*}
\end{proof}
\begin{lemma}\label{M:Vector:Value}
    Given position parameter \(\tensor{\theta}{}\), position vector can be evaluated as the following.
    \[
        \psi_0^{-1}\colon\tensor{\theta}{}\mapsto\tensor{p}{}
        =\frac{1}{\lambda}
        \begin{pmatrix}
            \prod_{j\in\Set{1..n}}{\cos_\lambda{\tensor{\theta}{^j}}}                                      \\
            \sin_\lambda{\tensor{\theta}{^{i-1}}}\prod_{j\in\Set{i..n}}{\cos_\lambda{\tensor{\theta}{^j}}} \\
            \sin_\lambda{{\tensor{\theta}{^n}}}                                                            \\
        \end{pmatrix}\text{.}
    \]
\end{lemma}
\begin{proof}[Proof of \cref{M:Vector:Value}]
    Simplify \cref{M:Vector:Position,M:Position}.
\end{proof}
\begin{lemma}\label{M:Parameter:Value}
    Given position vector \(\tensor{p}{}\), position parameter can be calculated as followed.
    \begin{align*}
        \psi_0
         & \colon\tensor{p}{}\mapsto\tensor{\theta}{}
        =
        \begin{pmatrix}
            \arctantwo_\lambda{\left(\tensor{p}{^2},\tensor{p}{^1}\right)}                                                           \\
            \arcsin_\lambda{\frac{\frac{1}{\lambda}\tensor{p}{^{i+1}}}{\prod_{j\in\Set{i+1..n}}{\cos_\lambda{\tensor{\theta}{^j}}}}} \\
            \arcsin_\lambda{\frac{1}{\lambda}\tensor{p}{^{n+1}}}                                                                     \\
        \end{pmatrix} \\
         & \in
        \begin{cases}
            P\to\left(-\frac{\pi}{\lambda},\frac{\pi}{\lambda}\right]\times\left[-\frac{1}{2}\frac{\pi}{\lambda},\frac{1}{2}\frac{\pi}{\lambda}\right]^{n-1} & \text{if \(\lambda>0\),} \\
            P\to\mathbb{R}^{n}                                                                                                                               & \text{otherwise.}        \\
        \end{cases}
    \end{align*}
\end{lemma}
\begin{proof}[Proof of \cref{M:Parameter:Value}]
    From \cref{M:Vector:Value},
    \begin{align*}
        \tensor{p}{}
                                              & =
        \frac{1}{\lambda}
        \begin{pmatrix}
            \prod_{j\in\Set{1..n}}{\cos_\lambda{\tensor{\theta}{^j}}}                                      \\
            \sin_\lambda{\tensor{\theta}{^{i-1}}}\prod_{j\in\Set{i..n}}{\cos_\lambda{\tensor{\theta}{^j}}} \\
            \sin_\lambda{{\tensor{\theta}{^n}}}                                                            \\
        \end{pmatrix}                                           \\
        \tensor{p}{^{n+1}}                    & = \frac{1}{\lambda}\sin_\lambda{\tensor{\theta}{^n}}                                                              \\
        \sin{\tensor{\theta}{^{n}}}           & = \lambda\tensor{p}{^{n+1}}                                                                                       \\
        \tensor{p}{^{i}}                      & = \frac{1}{\lambda}\sin_\lambda{\tensor{\theta}{^{i-1}}}\prod_{j\in\Set{i..n}}{\cos_\lambda{\tensor{\theta}{^j}}} \\
        \sin_\lambda{\tensor{\theta}{^{i-1}}} & = \frac{\lambda\tensor{p}{^{i}}}{\prod_{j\in\Set{i..n}}{\cos_\lambda{\tensor{\theta}{^j}}}}                       \\
        \sin_\lambda{\tensor{\theta}{^{1}}}   & = \frac{\lambda\tensor{p}{^2}}{\prod_{j\in\Set{2..n}}{\cos_\lambda{\tensor{\theta}{^j}}}}                         \\
        \tensor{p}{^1}                        & = \frac{1}{\lambda}\prod_{j\in\Set{1..n}}{\cos_\lambda{\tensor{\theta}{^j}}}                                      \\
        \cos_\lambda{\tensor{\theta}{^{1}}}   & = \frac{\lambda\tensor{p}{^1}}{\prod_{j\in\Set{2..n}}{\cos_\lambda{\tensor{\theta}{^j}}}}                         \\
        \theta                                & =
        \begin{pmatrix}
            \arctantwo_\lambda{\left(\tensor{p}{^2},\tensor{p}{^1}\right)}                                                 \\
            \arcsin_\lambda{\frac{\lambda\tensor{p}{^{i+1}}}{\prod_{j\in\Set{i+1..n}}{\cos_\lambda{\tensor{\theta}{^j}}}}} \\
            \arcsin_\lambda{\lambda\tensor{p}{^{n+1}}}                                                                     \\
        \end{pmatrix}
    \end{align*}
\end{proof}
\begin{lemma}\label{M:CoordinateChart}
    \[
        \Psi=\Set{\psi|
            \psi^{-1}
            \colon
            \tensor{\theta}{}\in \Theta^n
            \mapsto P_{\lambda,n}\left(\tensor{\theta}{}+\tensor{x}{}\right)
            \text{ for }
            \tensor{x}{}\in\mathbb{R}^n
        }
    \]
    is a coordinate chart of a \(C^\infty\) differential structure on the set of position matrix for
    \[
        \Theta =
        \begin{cases}
            \left(-\frac{1}{2}\frac{\pi}{\lambda},+\frac{1}{2}\frac{\pi}{\lambda}\right) & \text{if \(\lambda>0\),} \\
            \mathbb{R}                                                                   & \text{otherwise.}
        \end{cases}
    \]
\end{lemma}
\begin{proof}[Proof of \cref{M:CoordinateChart}]
    \begin{subproof}{\(\bigcup_{\psi\in\Psi} D_\psi\) covers the set of position matrix}
        Since there's a function from \(\tensor{\theta}{}\) to \(\tensor{P}{}\)
        and from \(\tensor{P}{}\) (to \(\tensor{p}{}\)) to \(\tensor{\theta}{}\),
        the mapping between \(\tensor{P}{}\) and \(\tensor{\theta}{}\) is bijective.
        \begin{align*}
            M\in \Set{P}
                    & \implies\exists\theta_0, M=P_{\lambda,n}\left(\theta_0\right)                   \\
                    & \implies\exists\theta_0, M=P_{\lambda,n}\left(0 +\theta_0\right)                \\
                    & \implies M\in R_{\psi^{-1}}                                                     \\
                    & \implies M\in D_{\psi}                                                          \\
                    & \implies M\in\bigcup_{\psi\in\Psi} D_\psi                                       \\
            \Set{P} & \subset\bigcup_{\psi\in\Psi}D_\psi                                              \\
            M\in\bigcup_{\psi\in\Psi} D_\psi
                    & \implies\exists\psi\in\Psi, M\in D_\psi                                         \\
                    & \implies\exists\psi\in\Psi, M\in R_{\psi^{-1}}                                  \\
                    & \implies\exists x_0\exists\theta\in S^n, M=P_{\lambda,n}\left(\theta+x_0\right) \\
                    & \implies\exists x_0, M=P_{\lambda,n}\left(0+x_0\right)                          \\
                    & \implies\exists x_0, M=P_{\lambda,n}\left(x_0\right)                            \\
                    & \implies M\in P                                                                 \\
            \Set{P} & \supset \bigcup_{\psi\in\Psi}D_\psi                                             \\
            \left[
                \Set{P} \subset \bigcup_{\psi\in\Psi}D_\psi
                \right]
            \wedge
            \left[
                \bigcup_{\psi\in\Psi}\Set{P} \supset \bigcup_{\psi\in\Psi}D_\psi
                \right]
                    & \implies \bigcup_{\psi\in\Psi}D_\psi = P
        \end{align*}
    \end{subproof}
    \begin{subproof}{every transition map is in differentability class \(C^\infty\)}
        Consider \(\psi_1,\psi_2\in\Psi\) and \(x_1, x_2\in\mathbb{R}^n\) where
        \begin{equation*}
            \psi_i^{-1}
            \in S^n\to R
            \colon\tensor{\theta}{}\mapsto P\left(\tensor{\theta}{}+x_i\right)
            \text{.}
        \end{equation*}

        Let \(\phi_i =\theta_i+x_i\).

        If \(\psi_1^{-1}\left(\theta_1\right)=\psi_2^{-1}\left(\theta_2\right)\),
        \begin{align*}
            \psi_1^{-1}\left(\theta_1\right)
             & =\psi_2^{-1}\left(\theta_2\right)                                                                               \\
            \psi_0^{-1}\left(\theta_1+x_1\right)
             & =\psi_0^{-1}\left(\theta_2+x_2\right)                                                                           \\
            \psi_0^{-1}\left(\phi_1\right)
             & =\psi_0^{-1}\left(\phi_2\right)                                                                                 \\
            \frac{1}{\lambda}
            \begin{pmatrix}
                \prod_{j\in\Set{1..n}}{\cos_\lambda{\tensor{{\phi_1}}{^j}}}                                        \\
                \sin_\lambda{\tensor{{\phi_1}}{^{i-1}}}\prod_{j\in\Set{i..n}}{\cos_\lambda{\tensor{{\phi_1}}{^j}}} \\
                \sin_\lambda{{\tensor{{\phi_1}}{^n}}}                                                              \\
            \end{pmatrix}
             & =
            \frac{1}{\lambda}
            \begin{pmatrix}
                \prod_{j\in\Set{1..n}}{\cos_\lambda{\tensor{{\phi_2}}{^j}}}                                        \\
                \sin_\lambda{\tensor{{\phi_2}}{^{i-1}}}\prod_{j\in\Set{i..n}}{\cos_\lambda{\tensor{{\phi_2}}{^j}}} \\
                \sin_\lambda{{\tensor{{\phi_2}}{^n}}}                                                              \\
            \end{pmatrix} \\
            \begin{pmatrix}
                \prod_{j\in\Set{1..n}}{\cos_\lambda{\tensor{{\phi_1}}{^j}}}                                        \\
                \sin_\lambda{\tensor{{\phi_1}}{^{i-1}}}\prod_{j\in\Set{i..n}}{\cos_\lambda{\tensor{{\phi_1}}{^j}}} \\
                \sin_\lambda{{\tensor{{\phi_1}}{^n}}}                                                              \\
            \end{pmatrix}
             & =
            \begin{pmatrix}
                \prod_{j\in\Set{1..n}}{\cos_\lambda{\tensor{{\phi_2}}{^j}}}                                        \\
                \sin_\lambda{\tensor{{\phi_2}}{^{i-1}}}\prod_{j\in\Set{i..n}}{\cos_\lambda{\tensor{{\phi_2}}{^j}}} \\
                \sin_\lambda{{\tensor{{\phi_2}}{^n}}}                                                              \\
            \end{pmatrix}
        \end{align*}
        If \(\lambda\ge 0\), by mathematical induction, \(\phi_1=m\frac{\pi}{\lambda}\pm\phi_2\).
        If \(\lambda\le 0\), by mathematical induction, \(\phi_1 =\phi_2\).
        Hence, the transition map \(\tau_{1,2} =\psi_2\circ\psi_1^{-1}\)
        is in the form of linear function \(\theta\mapsto c\pm\theta\)
        and is in differentability class \(C^\infty\).
    \end{subproof}
    Since the collection of mapping \(\Psi\) have the following properties
    \begin{APAenumerate}
        \item \(R_\psi\) is an open subset of real vector space (defined),
        \item \(\bigcup_{\psi\in\Psi} D_\psi\) covers the set of position matrix,
        \item transition map is in differentability class \(C^\infty\),
    \end{APAenumerate}
    the statement is proved by \cref{Manifold}.
\end{proof}
\begin{remark}\label{M:Manifold}
    Since the set of position matrices is a \(C^\infty\)-differentiable differential structure
    which can be acted on via the group action of the set of point matrices (matrix multiplication),
    it is a valid candidate for the manifold \(P\).
\end{remark}
\end{document}