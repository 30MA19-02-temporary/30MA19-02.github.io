% !TeX root = ../main.tex
\documentclass[../main.tex]{subfiles}
\begin{document}
\section{Model Foundation}
\subsection{Trigonometry}
\begin{definition}\label{M:Trigonometry}
\textit{Generalized trigonometric functions} \(f_\lambda\colon \R\to\R\) and \(f_\lambda^\ast:\R\to\R\) are defined as
\begin{align*}
f_\lambda\left(\theta\right)&\coloneqq
\begin{cases}
g\left(\lambda\theta\right)&\text{if \(\lambda\geq0\),}\\
h\left(\lambda\theta\right)&\text{otherwise,}\\
\end{cases}\\
f_\lambda^\ast\left(\theta\right)&\coloneqq
\begin{cases}
g\left(\lambda\theta\right)&\text{if \(\lambda\geq0\),}\\
h\left(-\lambda\theta\right)&\text{otherwise,}\\
\end{cases}
\end{align*}
where \(g\) (resp. \(h\)) are the associated trigonometric (resp. hyperbolic) function.
\end{definition}
\begin{example}[Generalized sine functions]\label{M:Trigonometry:Sine}
\begin{align*}
\sin_\lambda{\theta}
&\coloneqq
\begin{cases}
\sin\left(\lambda\theta\right)&\text{if \(\lambda\geq0\),}\\
\sinh\left(\lambda\theta\right)&\text{otherwise.}\\
\end{cases}\\
\sin_\lambda^\ast{\theta}
&\coloneqq
\begin{cases}
\sin\left(\lambda\theta\right)&\text{if \(\lambda\geq0\),}\\
\sinh\left(-\lambda\theta\right)&\text{otherwise.}\\
\end{cases}\\
&=
\begin{cases}
\sin\left(\lambda\theta\right)&\text{if \(\lambda\geq0\),}\\
-\sinh\left(\lambda\theta\right)&\text{otherwise.}\\
\end{cases}\\
&=\begin{cases}
\sin\left(\abs{\lambda}\theta\right)&\text{if \(\lambda\geq0\),}\\
\sinh\left(\abs{\lambda}\theta\right)&\text{otherwise.}\\
\end{cases}
\end{align*}
(see\cref{TrigonometrySinePlotted})
\end{example}
\begin{example}[Generalized cosine functions]\label{M:Trigonometry:Cosine}
\begin{align*}
\cos_\lambda{\theta}
&\coloneqq
\begin{cases}
\cos\left(\lambda\theta\right)&\text{if \(\lambda\geq0\),}\\
\cosh\left(\lambda\theta\right)&\text{otherwise.}\\
\end{cases}\\
\cos_\lambda^\ast{\theta}
&\coloneqq
\begin{cases}
\cos\left(\lambda\theta\right)&\text{if \(\lambda\geq0\),}\\
\cosh\left(-\lambda\theta\right)&\text{otherwise.}\\
\end{cases}\\
&=\begin{cases}
\cos\left(\lambda\theta\right)&\text{if \(\lambda\geq0\),}\\
\cosh\left(\lambda\theta\right)&\text{otherwise.}\\
\end{cases}\\
&=\cos_\lambda{\theta}
\end{align*}
(see\cref{TrigonometryCosinePlotted})
\end{example}
\begin{theorem}[Pythagorean's identity equivalence]\label{M:Trigonometry:Pythagorean}
\begin{align*}
\cos_\lambda^2{\theta}+\sign{\lambda}\sin_\lambda^2{\theta}&=1\\
\sec_\lambda^2{\theta}-\sign{\lambda}\tan_\lambda^2{\theta}&=1\\
\end{align*}
\end{theorem}
\begin{proof}[\proofof{M:Trigonometry:Pythagorean}]
Proof by exhaustion.
\end{proof}
\begin{proposition}[Generalized trigonometric functions of sum of arguments]\label{M:Trigonometry:Sum}
\begin{align*}
\sin_\lambda\left(\theta+\phi\right)
&=\sin_\lambda{\theta}\cos_\lambda{\phi}+\cos_\lambda{\theta}\sin_\lambda{\phi}\\
\sin_\lambda^\ast\left(\theta+\phi\right)
&=\sin_\lambda^\ast{\theta}\cos_\lambda{\phi}+\cos_\lambda{\theta}\sin_\lambda^\ast{\phi}\\
\cos_\lambda\left(\theta+\phi\right)
&=\cos_\lambda{\theta}\cos_\lambda{\phi}-\sign{\lambda}\sin_\lambda{\theta}\sin_\lambda{\phi}\\
&=\cos_\lambda{\theta}\cos_\lambda{\phi}-\sin_\lambda^\ast{\theta}\sin_\lambda{\phi}\\
&=\cos_\lambda{\theta}\cos_\lambda{\phi}-\sin_\lambda{\theta}\sin_\lambda^\ast{\phi}
\end{align*}
\end{proposition}
\begin{proof}[\proofof{M:Trigonometry:Sum}]
Proof by exhaustion.
\end{proof}
\begin{proposition}[Derivative of generalized trigonometric functions]\label{M:Trigonometry:Derivative}
\begin{align*}
{\sin_\lambda}^\prime{\theta}&=\lambda\cos_\lambda{\theta}\\
{\sin_\lambda^\ast}^\prime{\theta}&=\abs{\lambda}\cos_\lambda{\theta}\\
{\cos_\lambda}^\prime{\theta}&=-\lambda\sin_\lambda^\ast{\theta}\\
{\tan_\lambda}^\prime{\theta}&=\lambda\sec_\lambda^2{\theta}\\
{\tan_\lambda^\ast}^\prime{\theta}&=\abs{\lambda}\sec_\lambda^2{\theta}
\end{align*}
\end{proposition}
\begin{proof}[\proofof{M:Trigonometry:Derivative}]
Proof by exhaustion.
\end{proof}
\subsection{Matrices}
\begin{definition}\label{M:Rotation}
\textit{Generalized rotation matrix} is defined as
\begin{align*}
R_\lambda\left(\theta\right)&\coloneqq
\begin{bmatrix}
\cos_\lambda{\theta}&-\sin_\lambda^\ast{\theta}\\
\sin_\lambda{\theta}&\cos_\lambda{\theta}\\
\end{bmatrix}\text{,}
\end{align*}
where \(\theta\in\R\).
\end{definition}
\begin{corollary}[Generalized rotation matrix at zero]\label{M:Rotation:Identity}
\[
R_\lambda\left(0\right)
=
I_2
\]
\end{corollary}
\begin{proof}[\proofof{M:Rotation:Identity}]
Obvious
\end{proof}
\begin{corollary}[Generalized rotation matrix of sum of arguments]\label{M:Rotation:Sum}
\[
R_\lambda\left(\theta\right)R_\lambda\left(\phi\right)=R_\lambda\left(\theta+\phi\right)
\]
\end{corollary}
\begin{proof}[\proofof{M:Rotation:Sum}]
\begin{align*}
R_\lambda\left(\theta\right)R_\lambda\left(\phi\right)
&=\begin{bmatrix}
\cos_\lambda{\theta}&-\sin_\lambda^\ast{\theta}\\
\sin_\lambda{\theta}&\cos_\lambda{\theta}\\
\end{bmatrix}
\begin{bmatrix}
\cos_\lambda{\phi}&-\sin_\lambda^\ast{\phi}\\
\sin_\lambda{\phi}&\cos_\lambda{\phi}\\
\end{bmatrix}
&\text{(\cref{M:Rotation})}\\
&=\begin{bmatrix}
\cos_\lambda{\theta}\cos_\lambda{\phi}+\left(-\sin_\lambda^\ast{\theta}\right)\sin_\lambda{\phi}&
\cos_\lambda{\theta}\left(-\sin_\lambda^\ast{\phi}\right)+\left(-\sin_\lambda^\ast{\theta}\right)\cos_\lambda{\phi}\\
\sin_\lambda{\theta}\cos_\lambda{\phi}+\cos_\lambda{\theta}\sin_\lambda{\phi}&
\sin_\lambda{\theta}\left(-\sin_\lambda^\ast{\phi}\right)+\cos_\lambda{\theta}\cos_\lambda{\phi}\\
\end{bmatrix}
&\text{(\cref{Matrix:Product})}\\
&=\begin{bmatrix}
\cos_\lambda{\theta}\cos_\lambda{\phi}-\sin_\lambda^\ast{\theta}\sin_\lambda{\phi}&
-\left(\sin_\lambda^\ast{\theta}\cos_\lambda{\phi}+\cos_\lambda{\theta}\sin_\lambda^\ast{\phi}\right)\\
\sin_\lambda{\theta}\cos_\lambda{\phi}+\cos_\lambda{\theta}\sin_\lambda{\phi}&
\cos_\lambda{\theta}\cos_\lambda{\phi}-\sin_\lambda{\theta}\sin_\lambda^\ast{\phi}\\
\end{bmatrix}
&\text{(simplify)}\\
&=\begin{bmatrix}
\cos_\lambda{\theta+\phi}&-\sin_\lambda^\ast{\theta+\phi}\\
\sin_\lambda{\theta+\phi}&\cos_\lambda{\theta+\phi}\\
\end{bmatrix}
&\text{(\cref{M:Trigonometry:Sum})}\\
&=R_\lambda\left(\theta+\phi\right)
&\text{(\cref{M:Rotation})}\\
R_\lambda\left(\theta\right)
R_\lambda\left(\phi\right)
&=R_\lambda\left(\theta+\phi\right)
&\qedhere
\end{align*}
\end{proof}
\begin{corollary}[Inverse of generalized rotation matrix]\label{M:Rotation:Inverse}
\[
R_\lambda\left(\theta\right)^{-1}=R_\lambda\left(-\theta\right)
\]
\end{corollary}
\begin{proof}[\proofof{M:Rotation:Inverse}]
\begin{align*}
R_\lambda\left(\theta\right)
R_\lambda\left(-\theta\right)
&=R_\lambda\left(0\right)&\text{(\cref{M:Rotation:Sum})}\\
&=I_2&\text{(\cref{M:Rotation:Identity})}\\
R_\lambda\left(-\theta\right)
R_\lambda\left(\theta\right)
&=R_\lambda\left(0\right)&\text{(\cref{M:Rotation:Sum})}\\
&=I_2&\text{(\cref{M:Rotation:Identity})}
\end{align*}
\[
R_\lambda\left(\theta\right)R_\lambda\left(-\theta\right)
=R_\lambda\left(-\theta\right)R_\lambda\left(\theta\right)
=I_2
\]
\[
{R_\lambda\left(\theta\right)}^{-1}
=
R_\lambda\left(-\theta\right)
\qedhere
\]
\end{proof}
\begin{definition}\label{M:Position}
\textit{Position matrix} is defined recursively as
\begin{align*}
P_{\lambda,n}\left(\left\{\tensor{\theta}{^1},\dots,\tensor{\theta}{^n}\right\}\right)
&\coloneqq
\begin{bmatrix}
P_{\lambda,n-1}\left(\left\{\tensor{\theta}{^1},\dots,\tensor{\theta}{^{n-1}}\right\}\right)&0_{n\times 1}\\
0_{1\times n}&1\\
\end{bmatrix}
T_{2,n+1}
\begin{bmatrix}
R_\lambda\left(\tensor{\theta}{^n}\right)&0_{{2}\times{n-1}}\\
0_{{n-1}\times{2}}&{I}_{n-1}\\
\end{bmatrix}
T_{2,n+1}\text{,}\\
P_{\lambda,0}
\lambda&\coloneqq I_1\text{,}
\end{align*}
where \(\theta=\left\{\tensor{\theta}{^i}\right\}\in\R^n\) for \(i\in\range{1}{n}\).
\end{definition}
\begin{definition}\label{M:Position:Set}
Let \(P\left(n,\lambda\right)\) be set of position matrices.
\end{definition}
\begin{corollary}[Position matrix at zero]\label{M:Position:Set:Identity}
\[
P_{\lambda,n}
\left(0_{n}\right)
=
I_{n+1}
\]
\end{corollary}
\begin{proof}[\proofof{M:Position:Set:Identity}]
Prove by mathematical induction on \(n\),
Let
\begin{equation}\label{M:Position:Set:Identity:Proof:Induction}
P_{\lambda,n-1}\left(0_{n-1}\right)=I_n
\end{equation}
\begin{align*}
P_{\lambda, n}\left(0_{n}\right)
&=
\begin{bmatrix}
P_{\lambda,n-1}\left(0_{n-1}\right)&0_{n\times 1}\\
0_{1\times n}&1\\
\end{bmatrix}
T_{2, n+1}
\begin{bmatrix}
R_{\lambda}\left(0\right)&0_{{2}\times{n-1}}\\
0_{{n-1}\times{2}}&{I}_{n-1}\\
\end{bmatrix}
T_{2, n+1}
&\text{(\cref{M:Position})}\\
&=
\begin{bmatrix}
I_{n}&0_{n\times 1}\\
0_{1\times n}&1\\
\end{bmatrix}
T_{2, n+1}
\begin{bmatrix}
R_{\lambda}\left(0\right)&0_{{2}\times{n-1}}\\
0_{{n-1}\times{2}}&{I}_{n-1}\\
\end{bmatrix}
T_{2, n+1}
&\text{(\cref{M:Position:Set:Identity:Proof:Induction})}\\
&=
\begin{bmatrix}
I_{n}&0_{n\times 1}\\
0_{1\times n}&1\\
\end{bmatrix}
T_{2, n+1}
\begin{bmatrix}
I_{2}&0_{{2}\times{n-1}}\\
0_{{n-1}\times{2}}&{I}_{n-1}\\
\end{bmatrix}
T_{2, n+1}
&\text{(\cref{M:Rotation:Identity})}\\
&=
I_{n+1}
T_{2, n+1}
I_{n+1}
T_{2, n+1}
&\text{(\cref{Matrix:Identity:Block})}\\
&=
T_{2, n+1}
T_{2, n+1}
&\text{(\cref{Matrix:Identity})}\\
&=
I_{n+1}
&\text{(\cref{Matrix:Permutation:Square})}
\end{align*}
\[
P_{\lambda,n}
\left(0_{n}\right)
=
I_{n+1}
\qedhere
\]
\end{proof}
\begin{definition}\label{M:Orientation}
\textit{Orientation matrix} is defined as
\begin{align*}
Q^{\pm}_n\left(\phi_{n-1},\phi_{n-2},\dots,\phi_1\right)
&\coloneqq
\begin{bmatrix}
1&0_{1\times n}\\
0_{n\times 1}&X^{\pm}_{+1,n-1}\left(\phi_{n-1},\phi_{n-2},\dots,\phi_1\right)\\
\end{bmatrix}\text{,}\\
Q^{\pm}_0
&\coloneqq\pm I_1\text{,}
\end{align*}
where \(\phi_m\in\R^m\text{for } m\in\range{1}{n-1}\).
\end{definition}
\begin{definition}\label{M:Orientation:Set}
Let \(Q\left(n\right)\) be set of orientation matrices.
\end{definition}
\begin{corollary}[Orientation matrix at zero]\label{M:Orientation:Set:Identity}
\[
Q^{+}_{n}
\left(0_{n-1}, 0_{n-2},\dots\right)
=
I_{n+1}
\]
\end{corollary}
\begin{proof}[\proofof{M:Orientation:Set:Identity}]
Prove by mathematical induction on \(n\),
Let
\begin{equation}\label{M:Orientation:Set:Identity:Proof:Induction}
Q^{+}_{n-1}\left(0_{n-2}, 0_{n-3},\dots\right)=I_n
\end{equation}
\begin{align*}
Q^{+}_n\left(0_{n-1}, 0_{n-2},\dots\right)
&=
\begin{bmatrix}
1&0_{1\times n}\\
0_{n\times 1}&X^{\pm}_{+1,n-1}\left(0_{n-1},0_{n-2},\dots\right)\\
\end{bmatrix}
&\text{(\cref{M:Orientation})}\\
&=
\begin{bmatrix}
1&0_{1\times n}\\
0_{n\times 1}&P_{+1,n-1}\left(0_{n-1}\right)Q^{+}_{n-1}\left(0_{n-2}, 0_{n-3},\dots\right)\\
\end{bmatrix}
&\text{(\cref{M:Point})}\\
&=
\begin{bmatrix}
1&0_{1\times n}\\
0_{n\times 1}&P_{+1,n-1}\left(0_{n-1}\right) I_n\\
\end{bmatrix}
&\text{(\cref{M:Orientation:Set:Identity:Proof:Induction})}\\
&=
\begin{bmatrix}
1&0_{1\times n}\\
0_{n\times 1}&P_{+1,n-1}\left(0_{n-1}\right)\\
\end{bmatrix}
&\text{(\cref{Matrix:Identity})}\\
&=
\begin{bmatrix}
1&0_{1\times n}\\
0_{n\times 1}&I_{n}\\
\end{bmatrix}
&\text{(\cref{M:Position:Set:Identity})}\\
&=
I_{n+1}
&\text{(\cref{Matrix:Identity:Block})}
\end{align*}
\[
Q^{+}_n
\left(0_{n-1}, 0_{n-2},\dots\right)
=
I_{n+1}
\qedhere
\]
\end{proof}
\begin{definition}\label{M:Point}
\textit{Point matrix} is defined as
\begin{align*}
X^{\pm}_{\lambda,n}\left(\theta,\phi_{n-1},\phi_{n-2},\dots,\phi_1\right)&\coloneqq
P_{\lambda,n}\left(\theta\right)
Q^{\pm}_n\left(\phi_{n-1},\phi_{n-2},\dots,\phi_1\right)\text{,}
\end{align*}
where \(\theta\in\R^n\) and \(\phi_m\in\R^m\text{for } m\in\range{1}{n-1}\).
\end{definition}
\begin{definition}\label{M:Point:Set}
Let \(X\left(n,\lambda\right)\) be set of point matrices.
\end{definition}
\begin{corollary}[Position matrix as subset of point matrix]\label{M:Point:Position}
\[
P\left(\lambda,n\right)\subset X\left(\lambda,n\right)
\]
\end{corollary}
\begin{proof}[\proofof{M:Point:Position}]
\begin{align*}
\forall{P\in P\left(\lambda,n\right)}
\forall{Q\in Q\left(n\right)},
&P Q\in X\left(\lambda,n\right)
&\text{(\cref{M:Point})}\\
\implies
&P I\in X\left(\lambda,n\right)
&\text{(\cref{M:Orientation:Set:Identity})}\\
\implies
&P\in X\left(\lambda,n\right)
&\text{(\cref{Matrix:Identity})}
\end{align*}
\[
P\left(\lambda,n\right)\subset X\left(\lambda,n\right)\qedhere
\]
\end{proof}
\begin{corollary}[Point matrix at zero]\label{M:Point:Set:Identity}
\[
X^{+}_{\lambda,n}
\left(0_{n}, 0_{n-1},\dots\right)
=
I_{n+1}
\]
\end{corollary}
\begin{proof}[\proofof{M:Point:Set:Identity}]
It can be implied from\cref{M:Position:Set:Identity,M:Orientation:Set:Identity}.
\end{proof}
\subsection{Group structure}
\begin{proposition}
Group of position matrix with multiplication is a subgroup of an orthogonal group for \(\lambda>0\).
\[
\left(P\left(n,\lambda\right),\cdot\right)\cong O\left(n+1\right)
\]
\end{proposition}
\begin{proof}
\begin{align*}
P&=P_{\lambda,n}\left(\left\{\tensor{\theta}{^1},\dots,\tensor{\theta}{^n}\right\}\right)\\
&\coloneqq
\begin{bmatrix}
P_{\lambda,n-1}\left(\left\{\tensor{\theta}{^1_{n}},\dots,\tensor{\theta}{^{n-1}}\right\}\right)&0_{n\times 1}\\
0_{1\times n}&1\\
\end{bmatrix}
T_{2,n+1}
\begin{bmatrix}
R_\lambda\left(\tensor{\theta}{^n}\right)&0_{{2}\times{n-1}}\\
0_{{n-1}\times{2}}&{I}_{n-1}\\
\end{bmatrix}
T_{2,n+1}
&&\text{(\cref{M:Position})}\\
P^T
&=
T_{2,n+1}^T
\begin{bmatrix}
R_\lambda\left(\tensor{\theta}{^n}\right)&0_{{2}\times{n-1}}\\
0_{{n-1}\times{2}}&{I}_{n-1}\\
\end{bmatrix}^T
T_{2,n+1}^T
\begin{bmatrix}
P_{\lambda,n-1}\left(\left\{\tensor{\theta}{^1_{n}},\dots,\tensor{\theta}{^{n-1}}\right\}\right)&0_{n\times 1}\\
0_{1\times n}&1\\
\end{bmatrix}^T
&&\text{(Transpose of product)}\\
&=
T_{2,n+1}^T
\begin{bmatrix}
R_\lambda\left(\tensor{\theta}{^n}\right)^T&0_{{2}\times{n-1}}\\
0_{{n-1}\times{2}}&{I}_{n-1}\\
\end{bmatrix}
T_{2,n+1}^T
\begin{bmatrix}
P_{\lambda,n-1}\left(\left\{\tensor{\theta}{^1_{n}},\dots,\tensor{\theta}{^{n-1}}\right\}\right)^T&0_{n\times 1}\\
0_{1\times n}&1\\
\end{bmatrix}
&&\text{(Transpose of block)}\\
&=
T_{2,n+1}
\begin{bmatrix}
R_\lambda\left(\tensor{\theta}{^n}\right)^T&0_{{2}\times{n-1}}\\
0_{{n-1}\times{2}}&{I}_{n-1}\\
\end{bmatrix}
T_{2,n+1}
\begin{bmatrix}
P_{\lambda,n-1}\left(\left\{\tensor{\theta}{^1_{n}},\dots,\tensor{\theta}{^{n-1}}\right\}\right)^T&0_{n\times 1}\\
0_{1\times n}&1\\
\end{bmatrix}
&&\text{(Transpose of permutation matrix)}\\
PP^T
&=
\left(\begin{bmatrix}
P_{\lambda,n-1}\left(\left\{\tensor{\theta}{^1_{n}},\dots,\tensor{\theta}{^{n-1}}\right\}\right)&0_{n\times 1}\\
0_{1\times n}&1\\
\end{bmatrix}
T_{2,n+1}
\begin{bmatrix}
R_\lambda\left(\tensor{\theta}{^n}\right)&0_{{2}\times{n-1}}\\
0_{{n-1}\times{2}}&{I}_{n-1}\\
\end{bmatrix}
T_{2,n+1}\right)\\
&\left(T_{2,n+1}
\begin{bmatrix}
R_\lambda\left(\tensor{\theta}{^n}\right)^T&0_{{2}\times{n-1}}\\
0_{{n-1}\times{2}}&{I}_{n-1}\\
\end{bmatrix}
T_{2,n+1}
\begin{bmatrix}
P_{\lambda,n-1}\left(\left\{\tensor{\theta}{^1_{n}},\dots,\tensor{\theta}{^{n-1}}\right\}\right)^T&0_{n\times 1}\\
0_{1\times n}&1\\
\end{bmatrix}\right)
&&\text{(equation above)}\\
&=
\begin{bmatrix}
P_{\lambda,n-1}\left(\left\{\tensor{\theta}{^1_{n}},\dots,\tensor{\theta}{^{n-1}}\right\}\right)&0_{n\times 1}\\
0_{1\times n}&1\\
\end{bmatrix}
T_{2,n+1}
\begin{bmatrix}
R_\lambda\left(\tensor{\theta}{^n}\right)&0_{{2}\times{n-1}}\\
0_{{n-1}\times{2}}&{I}_{n-1}\\
\end{bmatrix}\\
&
\begin{bmatrix}
R_\lambda\left(\tensor{\theta}{^n}\right)^T&0_{{2}\times{n-1}}\\
0_{{n-1}\times{2}}&{I}_{n-1}\\
\end{bmatrix}
T_{2,n+1}
\begin{bmatrix}
P_{\lambda,n-1}\left(\left\{\tensor{\theta}{^1_{n}},\dots,\tensor{\theta}{^{n-1}}\right\}\right)^T&0_{n\times 1}\\
0_{1\times n}&1\\
\end{bmatrix}
&&\text{()}\\
&=
\begin{bmatrix}
P_{\lambda,n-1}\left(\left\{\tensor{\theta}{^1_{n}},\dots,\tensor{\theta}{^{n-1}}\right\}\right)&0_{n\times 1}\\
0_{1\times n}&1\\
\end{bmatrix}
T_{2,n+1}
\begin{bmatrix}
R_\lambda\left(\tensor{\theta}{^n}\right)R_\lambda\left(\tensor{\theta}{^n}\right)^T&0_{{2}\times{n-1}}\\
0_{{n-1}\times{2}}&{I}_{n-1}\\
\end{bmatrix}\\
&
T_{2,n+1}
\begin{bmatrix}
P_{\lambda,n-1}\left(\left\{\tensor{\theta}{^1_{n}},\dots,\tensor{\theta}{^{n-1}}\right\}\right)^T&0_{n\times 1}\\
0_{1\times n}&1\\
\end{bmatrix}&&\text{(\cref{Matrix:Product:Block})}\\
&=
\begin{bmatrix}
P_{\lambda,n-1}\left(\left\{\tensor{\theta}{^1_{n}},\dots,\tensor{\theta}{^{n-1}}\right\}\right)&0_{n\times 1}\\
0_{1\times n}&1\\
\end{bmatrix}
T_{2,n+1}\\
&
T_{2,n+1}
\begin{bmatrix}
P_{\lambda,n-1}\left(\left\{\tensor{\theta}{^1_{n}},\dots,\tensor{\theta}{^{n-1}}\right\}\right)^T&0_{n\times 1}\\
0_{1\times n}&1\\
\end{bmatrix}
&&\text{()}\\
&=
\begin{bmatrix}
P_{\lambda,n-1}\left(\left\{\tensor{\theta}{^1_{n}},\dots,\tensor{\theta}{^{n-1}}\right\}\right)&0_{n\times 1}\\
0_{1\times n}&1\\
\end{bmatrix}
\begin{bmatrix}
P_{\lambda,n-1}\left(\left\{\tensor{\theta}{^1_{n}},\dots,\tensor{\theta}{^{n-1}}\right\}\right)^T&0_{n\times 1}\\
0_{1\times n}&1\\
\end{bmatrix}&&\text{()}\\
&=
\begin{bmatrix}
P_{\lambda,n-1}\left(\left\{\tensor{\theta}{^1_{n}},\dots,\tensor{\theta}{^{n-1}}\right\}\right)P_{\lambda,n-1}\left(\left\{\tensor{\theta}{^1_{n}},\dots,\tensor{\theta}{^{n-1}}\right\}\right)^T&0_{n\times 1}\\
0_{1\times n}&1\\
\end{bmatrix}&&\text{(\cref{Matrix:Product:Block})}\\
&=I_{n+1}&&\text{(mathematical induction)}
\end{align*}
\end{proof}
\begin{proposition}
Group of position matrix with multiplication is a subgroup of an \(\left(1,n\right)\)-orthochronus indefinite orthogonal group for \(\lambda<0\).
\[
\left(P\left(n,\lambda\right),\cdot\right)\cong O_+\left(n,1\right)
\]
\end{proposition}
\begin{proof}
\begin{align*}
P
&=P_{\lambda,n}\left(\left\{\tensor{\theta}{^1},\dots,\tensor{\theta}{^n}\right\}\right)\\
&\coloneqq
\begin{bmatrix}
P_{\lambda,n-1}\left(\left\{\tensor{\theta}{^1_{n}},\dots,\tensor{\theta}{^{n-1}}\right\}\right)&0_{n\times 1}\\
0_{1\times n}&1\\
\end{bmatrix}
T_{2,n+1}
\begin{bmatrix}
R_\lambda\left(\tensor{\theta}{^n}\right)&0_{{2}\times{n-1}}\\
0_{{n-1}\times{2}}&{I}_{n-1}\\
\end{bmatrix}
T_{2,n+1}&&\text{(\cref{M:Position})}\\
P^T
&=
T_{2,n+1}^T
\begin{bmatrix}
R_\lambda\left(\tensor{\theta}{^n}\right)&0_{{2}\times{n-1}}\\
0_{{n-1}\times{2}}&{I}_{n-1}\\
\end{bmatrix}^T
T_{2,n+1}^T
\begin{bmatrix}
P_{\lambda,n-1}\left(\left\{\tensor{\theta}{^1_{n}},\dots,\tensor{\theta}{^{n-1}}\right\}\right)&0_{n\times 1}\\
0_{1\times n}&1\\
\end{bmatrix}^T&&\text{(Transpose of product)}\\
&=
T_{2,n+1}^T
\begin{bmatrix}
R_\lambda\left(\tensor{\theta}{^n}\right)^T&0_{{2}\times{n-1}}\\
0_{{n-1}\times{2}}&{I}_{n-1}\\
\end{bmatrix}
T_{2,n+1}^T
\begin{bmatrix}
P_{\lambda,n-1}\left(\left\{\tensor{\theta}{^1_{n}},\dots,\tensor{\theta}{^{n-1}}\right\}\right)^T&0_{n\times 1}\\
0_{1\times n}&1\\
\end{bmatrix}
&&\text{(Transpose of block)}\\
&=
T_{2,n+1}
\begin{bmatrix}
R_\lambda\left(\tensor{\theta}{^n}\right)^T&0_{{2}\times{n-1}}\\
0_{{n-1}\times{2}}&{I}_{n-1}\\
\end{bmatrix}
T_{2,n+1}
\begin{bmatrix}
P_{\lambda,n-1}\left(\left\{\tensor{\theta}{^1_{n}},\dots,\tensor{\theta}{^{n-1}}\right\}\right)^T&0_{n\times 1}\\
0_{1\times n}&1\\
\end{bmatrix}
&&\text{(Transpose of permutation matrix)}\\
g
&=\diag{-1,1,\dots,1}\\
gPgP^T
&=I_{n+1}&&\text{(mathematical induction)}
\end{align*}
\begin{align*}
\tensor{P}{^1_1}&> 0
\end{align*}
\end{proof}
\begin{proposition}
Group of position matrix with multiplication is isomorphic to translation group for \(\lambda\to0\).
\[
\left(P\left(n,\lambda\right),\cdot\right)\cong E\left(n\right)
\]
\end{proposition}
\begin{proof}
\begin{align*}
R_\lambda\left(\theta\right)
&\coloneqq\begin{bmatrix}
\cos_\lambda\left(\theta\right)&-\sin_\lambda^\ast\left(\theta\right)\\
\sin_\lambda\left(\theta\right)&\cos_\lambda\left(\theta\right)\\
\end{bmatrix}&&\text{(\cref{M:Rotation})}\\
&\to\begin{bmatrix}
1&0\\
\lambda\theta&1\\
\end{bmatrix}&&\text{(Limits of the functions)}\\
P_{\lambda,n}\left(\left\{\tensor{\theta}{^1},\dots,\tensor{\theta}{^n}\right\}\right)
&\coloneqq
\begin{bmatrix}
P_{\lambda,n-1}\left(\left\{\tensor{\theta}{^1_{n}},\dots,\tensor{\theta}{^{n-1}}\right\}\right)&0_{n\times 1}\\
0_{1\times n}&1\\
\end{bmatrix}
T_{2,n+1}
\begin{bmatrix}
R_\lambda\left(\tensor{\theta}{^n}\right)&0_{{2}\times{n-1}}\\
0_{{n-1}\times{2}}&{I}_{n-1}\\
\end{bmatrix}
T_{2,n+1}&&\text{(\cref{M:Position})}\\
&\to
\begin{bmatrix}
P_{\lambda,n-1}\left(\left\{\tensor{\theta}{^1_{n}},\dots,\tensor{\theta}{^{n-1}}\right\}\right)&0_{n\times 1}\\
0_{1\times n}&1\\
\end{bmatrix}
T_{2,n+1}
\begin{bmatrix}
\begin{matrix*}1&0\\
\lambda\tensor{\theta}{^n}&0\\
\end{matrix*}&0_{{2}\times{n-1}}\\
0_{{n-1}\times{2}}&{I}_{n-1}\\
\end{bmatrix}
T_{2,n+1}&&\text{(equation above)}\\
&=
\begin{bmatrix}
P_{\lambda,n-1}\left(\left\{\tensor{\theta}{^1_{n}},\dots,\tensor{\theta}{^{n-1}}\right\}\right)&0_{n\times 1}\\
0_{1\times n}&1\\
\end{bmatrix}
\begin{bmatrix}
{I}_{n}&0_{n\times1}\\
\begin{matrix*}\lambda\tensor{\theta}{^n}&0_{1\times n-1}\end{matrix*}&1\\
\end{bmatrix}&&\text{(property of permutation matrix)}\\
&=
\begin{bmatrix}
P_{\lambda,n-1}\left(\left\{\tensor{\theta}{^1_{n}},\dots,\tensor{\theta}{^{n-1}}\right\}\right)&0_{n\times1}\\
\begin{matrix*}\lambda\tensor{\theta}{^n}&0_{1\times n-1}\end{matrix*}&1\\
\end{bmatrix}&&\text{(\cref{Matrix:Product:Block})}\\
P_{\lambda,n}\left(\tensor{\theta}{}\right)
&\to
\begin{bmatrix}
1&0_{n\times1}\\
\lambda\tensor{\theta}{}&I_n\\
\end{bmatrix}&&\text{(mathematical induction)}\\
P_{\lambda,n}\left(\tensor{\theta}{}\right) P_{\lambda,n}\left(\tensor{\phi}{}\right)
&\to
\begin{bmatrix}
1&0_{n\times1}\\
\lambda\tensor{\theta}{}&I_n\\
\end{bmatrix}\begin{bmatrix}
1&0_{n\times1}\\
\lambda\tensor{\phi}{}&I_n\\
\end{bmatrix}&&\text{(equation above)}\\
&=
\begin{bmatrix}
1&0_{n\times1}\\
\lambda\tensor{\theta}{}+\lambda\tensor{\phi}{}&I_n\\
\end{bmatrix}&&\text{(\cref{Matrix:Product:Block})}
\end{align*}
\end{proof}
\begin{proposition}
Group of orienatation with multiplication is isomorphic to orthogonal group.
\[
\left(Q\left(n\right),\cdot\right)\cong O\left(n\right)
\]
\end{proposition}
\begin{proposition}
Group of point with multiplication is isomorphic to orthogonal group for \(\lambda>0\).
\[
\left(X\left(n\right),\cdot\right)\cong O\left(n\right)
\]
\end{proposition}
\begin{proposition}
Group of point with multiplication is isomorphic to \(\left(1,n\right)\)-orthochronus indefinite orthogonal group for \(\lambda<0\).
\[
\left(X\left(n\right),\cdot\right)\cong O_+\left(n,1\right)
\]
\end{proposition}
\begin{proposition}
Group of point with multiplication is isomorphic to Euclidean group for \(\lambda\to0\).
\[
\left(X\left(n\right),\cdot\right)\cong E\left(n\right)
\]
\end{proposition}
\end{document}